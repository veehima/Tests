\documentclass[10pt]{article}
\usepackage[paperwidth=20cm,paperheight=17cm,top=1.5cm,bottom=1.5cm,left=1cm,right=1cm]{geometry}
\usepackage{tikz}
\usepackage{fancybox,xspace,colortbl,calc,ifthen}
\usepackage{settobox}
\usepackage[utf8]{inputenc}
\usepackage[russian]{babel}
\usepackage{mathtext}
\usepackage{amssymb,amsmath,dejavu,pifont}
\usepackage[pdftex,unicode,pdfborder={0 0 0},pdfhighlight=/P,breaklinks]{hyperref}
\usepackage{eulervm}
\usepackage{flowfram}
\usepackage{caption}
\usepackage{floatrow}
\usepackage{picins}
\usepackage[filename=tooltipy,active]{fancytooltips}


\usetikzlibrary{shapes,snakes,arrows,backgrounds}

\newcommand{\sym}{\textcolor{red}{\ding{229}}}
\renewcommand{\baselinestretch}{1.25}\normalsize
\newcommand{\tr}{\ding{115}}
\newcommand{\pr}{Pr}

\pagestyle{empty}
\definecolor{p}{rgb}{1,0.874,0.666}
\definecolor{pex}{rgb}{0.686,0.898,1}
\newstaticframe{\paperwidth}{\paperheight}{-1cm}{-1.5cm}[bkr]
\setstaticcontents*{bkr}{\begin{tikzpicture}[ultra thick]
                            \shadedraw[left color = red!15,right color = red!50,draw=red!70]
                                        (0.001\paperwidth,0.001\paperheight) rectangle +(0.2\paperwidth,0.999\paperheight);
                            \draw[fill = p!50,draw=p]
                                        (0.202\paperwidth,0.001\paperheight) rectangle +(0.797\paperwidth,0.999\paperheight);
                        \end{tikzpicture}
}
\newflowframe{0.72\textwidth}{\textheight}{0.22\paperwidth}{0cm}
\newdynamicframe{3cm}{1cm}{-0.3cm}{10.8cm}[but1]
\newdynamicframe{3cm}{1cm}{-0.3cm}{9.7cm}[but2]
\newdynamicframe{1.5cm}{1cm}{0.35cm}{8.4cm}[but3]
\newdynamicframe{1.5cm}{1cm}{-0.475cm}{7.2cm}[but4]
\newdynamicframe{1.5cm}{1cm}{1.175cm}{7.2cm}[but5]
\newdynamicframe{1.5cm}{1cm}{0.35cm}{6cm}[but6]
\newdynamicframe{3cm}{1cm}{-0.3cm}{4.8cm}[but7]
\newdynamicframe{3cm}{1cm}{-0.3cm}{3.6cm}[but8]
\newdynamicframe{3cm}{1cm}{-0.3cm}{2.4cm}[but9]
\newdynamicframe{3cm}{1cm}{-0.3cm}{1.2cm}[but10]
\newdynamicframe{3cm}{1cm}{-0.3cm}{0cm}[but11]
\newdynamicframe{3cm}{1cm}{-0.05cm}{-1.5cm}[but12]

\setdynamiccontents*{but1}{
\href{mailto:amiheev@mail.ru}{\protect\parbox[c][0.9cm][t]{2.7cm}{
\begin{tikzpicture}[rounded corners=1ex,ultra thick]
\node[rectangle, shade, top color=blue!20, bottom color=blue!50, draw=blue!50,
minimum width=2.6cm, minimum height=0.8cm, text centered,font=\footnotesize](h1){Письмо автору};
\end{tikzpicture}}}}
\setdynamiccontents*{but2}{
\ifthenelse{\thepage=2}{\protect\parbox[c][0.9cm][t]{2.7cm}{
\begin{tikzpicture}[rounded corners=1ex,ultra thick]
\node[rectangle, shade, top color=blue!10, bottom color=blue!25, draw=blue!25,
minimum width=2.6cm, minimum height=0.8cm, text centered,font=\footnotesize,text=black!50](h2){Содержание};
\end{tikzpicture}}}{\hyperlink{cont}{\protect\parbox[c][0.9cm][t]{2.7cm}{
\begin{tikzpicture}[rounded corners=1ex,ultra thick]
\node[rectangle, shade, top color=blue!20, bottom color=blue!50, draw=blue!50,
minimum width=2.6cm, minimum height=0.8cm, text centered,font=\footnotesize](h2){Содержание};
\end{tikzpicture}}}}}
\setdynamiccontents*{but3}{
\ifthenelse{\thepage=1}{\protect\parbox[c][0.9cm][t]{1.4cm}{
\begin{tikzpicture}[rounded corners=1ex,ultra thick]
\node[rectangle, shade, top color=blue!10, bottom color=blue!25,
draw=blue!25, minimum width=1.3cm, minimum height=0.8cm, text
centered,font=\large,text=black!50](h3){\Large$\blacktriangleleft\!\blacktriangleleft$};
\end{tikzpicture}}}{\Acrobatmenu{FirstPage}{\protect\parbox[c][0.9cm][t]{1.4cm}{
\begin{tikzpicture}[rounded corners=1ex,ultra thick]
\node[rectangle, shade, top color=blue!20, bottom color=blue!50,
draw=blue!50, minimum width=1.3cm, minimum height=0.8cm, text
centered,font=\large](h3){\Large$\blacktriangleleft\!\blacktriangleleft$};
\end{tikzpicture}}}}}
\setdynamiccontents*{but4}{
\ifthenelse{\thepage=1}{\protect\parbox[c][0.9cm][t]{1.4cm}{
\begin{tikzpicture}[rounded corners=1ex,ultra thick]
\node[rectangle, shade, top color=blue!10, bottom color=blue!25,
draw=blue!25, minimum width=1.3cm, minimum height=0.8cm, text
centered,font=\large,text=black!50](h4){\Large$\blacktriangleleft$};
\end{tikzpicture}}}{\Acrobatmenu{PrevPage}{\protect\parbox[c][0.9cm][t]{1.4cm}{
\begin{tikzpicture}[rounded corners=1ex,ultra thick]
\node[rectangle, shade, top color=blue!20, bottom color=blue!50,
draw=blue!50, minimum width=1.3cm, minimum height=0.8cm, text
centered,font=\large](h4){\Large$\blacktriangleleft$};
\end{tikzpicture}}}}}
\setdynamiccontents*{but5}{
\ifthenelse{\thepage=\pageref{lastpage}}{\protect\parbox[c][0.9cm][t]{1.4cm}{
\begin{tikzpicture}[rounded corners=1ex,ultra thick]
\node[rectangle, shade, top color=blue!10, bottom color=blue!25,
draw=blue!25, minimum width=1.3cm, minimum height=0.8cm, text
centered,font=\large,text=black!50](h5){\Large$\blacktriangleright$};
\end{tikzpicture}}}{\Acrobatmenu{NextPage}{\protect\parbox[c][0.9cm][t]{1.4cm}{
\begin{tikzpicture}[rounded corners=1ex,ultra thick]
\node[rectangle, shade, top color=blue!20, bottom color=blue!50,
draw=blue!50, minimum width=1.3cm, minimum height=0.8cm, text
centered,font=\large](h5){\Large$\blacktriangleright$};
\end{tikzpicture}}}}}
\setdynamiccontents*{but6}{
\ifthenelse{\thepage=\pageref{lastpage}}{\protect\parbox[c][0.9cm][t]{1.4cm}{
\begin{tikzpicture}[rounded corners=1ex,ultra thick]
\node[rectangle, shade, top color=blue!10, bottom color=blue!25,
draw=blue!25, minimum width=1.3cm, minimum height=0.8cm, text
centered,font=\large,text=black!50](h6){\Large$\blacktriangleright\!\blacktriangleright$};
\end{tikzpicture}}}{\Acrobatmenu{LastPage}{\protect\parbox[c][0.9cm][t]{1.4cm}{
\begin{tikzpicture}[rounded corners=1ex,ultra thick]
\node[rectangle, shade, top color=blue!20, bottom color=blue!50,
draw=blue!50, minimum width=1.3cm, minimum height=0.8cm, text
centered,font=\large](h6){\Large$\blacktriangleright\!\blacktriangleright$};
\end{tikzpicture}}}}}
\setdynamiccontents*{but7}{
\Acrobatmenu{GoBack}{\protect\parbox[c][0.9cm][t]{2.7cm}{
\begin{tikzpicture}[rounded corners=1ex,ultra thick]
\node[rectangle, shade, top color=blue!20, bottom color=blue!50, draw=blue!50,
minimum width=2.6cm, minimum height=0.8cm, text centered,font=\footnotesize](h7) {Вернуться};
\end{tikzpicture}}}}
\setdynamiccontents*{but8}{
\Acrobatmenu{GoToPage}{\protect\parbox[c][0.9cm][t]{2.7cm}{
\begin{tikzpicture}[rounded corners=1ex,ultra thick]
\node[rectangle, shade, top color=blue!20, bottom color=blue!50,
draw=blue!50, minimum width=2.6cm, minimum height=0.8cm, text
centered,font=\footnotesize](h8) {На страницу\ldots};
\end{tikzpicture}}}}
\setdynamiccontents*{but9}{
\Acrobatmenu{FullScreen}{\protect\parbox[c][0.9cm][t]{2.7cm}{
\begin{tikzpicture}[rounded corners=1ex,ultra thick]
\node[rectangle, shade, top color=blue!20, bottom color=blue!50, draw=blue!50,
minimum width=2.6cm, minimum height=0.8cm, text centered,font=\footnotesize](h9){Во весь экран};
\end{tikzpicture}}}}
\setdynamiccontents*{but10}{
\Acrobatmenu{Close}{\protect\parbox[c][0.9cm][t]{2.7cm}{
\begin{tikzpicture}[rounded corners=1ex,ultra thick]
\node[rectangle, shade, top color=blue!20, bottom color=blue!50, draw=blue!50,
minimum width=2.6cm, minimum height=0.8cm, text centered,font=\footnotesize](h10){Закрыть};
\end{tikzpicture}}}}
\setdynamiccontents*{but11}{
\Acrobatmenu{Quit}{\protect\parbox[c][0.9cm][t]{2.7cm}{
\begin{tikzpicture}[rounded corners=1ex,ultra thick]
\node[rectangle, shade, top color=blue!20, bottom color=blue!50, draw=blue!50,
minimum width=2.6cm, minimum height=0.8cm, text centered,font=\footnotesize](h11){Выход};
\end{tikzpicture}}}}
\setdynamiccontents*{but12}{
Cтр.~\thepage~из~\pageref{lastpage}}

\newdynamicframe{3cm}{2cm}{-0.8cm}{13.3cm}[but13]
\setdynamiccontents*{but13}{
\includegraphics[width=0.2\textwidth]{logo}
}

\newenvironment{defnt}%
{\begin{center}\fboxsep=1.6\fboxrule \shadowsize=4pt\begin{Sbox}
\begin{minipage}[c]{0.58\textwidth}}%
{\end{minipage}\end{Sbox}\shadowbox{\fboxsep=5pt\colorbox[rgb]{1,0.725,0.474}{\TheSbox}}
\end{center}}

\newcounter{primer}
\numberwithin{primer}{section}
\renewcommand{\theprimer}{\thesection.\arabic{primer}}

\newenvironment{examp}%
{\refstepcounter{primer}
\begin{center}\fboxsep=1.6\fboxrule \shadowsize=4pt \begin{Sbox}
\begin{minipage}[c]{0.63\textwidth}\shadowsize=2pt\shadowbox{\fboxsep=2pt\colorbox[rgb]{0.843,0.98,1}{\large
Пример \theprimer}}
\par\vspace*{6pt}}%
{\end{minipage}\end{Sbox}\shadowbox{\fboxsep=5pt\colorbox[rgb]{0.686,0.898,1}{\TheSbox}}
\end{center}}

\newenvironment{exampdop}%
{%\renewcommand{\baselinestretch}{1}\normalsize
	\begin{center}\fboxsep=10pt \begin{Sbox}
			\begin{minipage}[c]{0.9\textwidth}\small \textbf{Пример \theprimer} (Продолжение)
				\par\vspace*{6pt}}%
			{\end{minipage}\end{Sbox}\doublebox{\fboxsep=5pt\TheSbox}
	\end{center}
	%\renewcommand{\baselinestretch}{1.25}\normalsize
}


\newenvironment{ist_fig}[2]%
{\begin{center}\fboxsep=1.6\fboxrule \shadowsize=4pt\begin{Sbox}
\begin{minipage}[c]{0.7\textwidth}
\parpic[r][t]{\includegraphics[width=0.25\textwidth]{#1}}
\picskip{#2}\small}%
{\end{minipage}\end{Sbox}\shadowbox{\fboxsep=5pt\colorbox[rgb]{1,0.988,0.737}{\TheSbox}}
\end{center}}


\newcommand{\term}[1]{\textcolor{red}{\emph{#1}}}

\addto{\captionsrussian}{\renewcommand{\abstractname}{Немного об
этой книге \dots}}
\renewcommand{\refname}{Список литературы}

\numberwithin{equation}{section}
\renewcommand{\theequation}{\thesection.\arabic{equation}}

\newcommand{\refform}[1]{\textcolor{red}{(\ref{#1})}}

\title{\bf ЛИНЕЙНАЯ И ВЕКТОРНАЯ АЛГЕБРА}
\author{\tooltip{Михеев~А.\,В.}{ya}}
\date{}

\frenchspacing
\begin{document}

\maketitle

\thispagestyle{empty}
\begin{abstract}
%Курс высшей математики для гуманитариев.
%
%
%Слово математика происходит от греческого
%$\mu\alpha\theta\eta\mu\alpha$ [матэма]
%--- <<знание>>, <<наука>>.
Основное предназначение этого электронного учебного пособия --- помочь студентам самостоятельно освоить программу раздела <<Линейная и векторная алгебра>> курса <<Математика>>. Это пособие ни в коем случае не может заменить подробные печатные курсы высшей математики. Главная цель, которую преследовал автор --- дать краткое и, в то же время полное, введение в основные понятия и задачи линейной и векторной алгебры. Пособие содержит много иллюстраций, примеров решения типовых задач, сведений из истории математики. 

Много интересной, но не очень важной информации вынесено в аннотации, чтобы их увидеть --- наведите курсор мыши на слово, отмеченное символом \includegraphics[width=0.3cm]{fancytipmark.pdf}, и <<щёлкните>> по нему. Слева --- панель навигации по документу. Смысл имеющихся там кнопочек не требует пояснений\dots 

Надеюсь, это пособие окажется полезным для Вас.
\end{abstract}\newpage
\hypertarget{cont}{\tableofcontents}\newpage
%\section{Линейная алгебра}
\section{Матрицы и действия над ними}

\begin{defnt}
Матрицей размерности $m\times n$ называется таблица чисел,
расположенных в определенном порядке. При этом $m$ задаeт число
строк в таблице, а $n$ --- число столбцов.
\end{defnt}


Числа, образующие матрицу, называются её элементами. Положение
каждого элемента однозначно определяется номером строки и столбца,
на пересечении которых он находится. Элементы матрицы обозначаются
символом $a_{ij}$, где $i$ --- номер строки, а $j$ --- номер
столбца.
\begin{equation}\label{eq:matrix}
A = \begin{pmatrix}
  a_{11} & a_{12} & \cdots & a_{1n} \\
  a_{21} & a_{22} & \cdots & a_{2n} \\
  \vdots & \vdots & \ddots & \vdots \\
  a_{m1} & a_{m2} & \cdots & a_{mn} \\
\end{pmatrix}.
\end{equation}
Как видим, номера строк в матрице возрастают снизу вверх, а номера
столбцов --- слева направо.

\begin{examp}\label{prim:matr}
Матрица $\begin{pmatrix}
  -1 & 7 & -5 & 9 & 10 \\
  2{,}1 &-5{,}06 & -12 & 0 & 4 \\
\end{pmatrix}$ имеет размерность $2\times5$, т.е. она
содержит 2 строки и 5 столбцов. $a_{14}=9$ --- элемент, стоящий на
пересечении 1--ой строки и 4--го столбца;  $a_{22}=-5{,}06$ ---
элемент, стоящий на пересечении 2--ой строки и 2--го столбца.
\end{examp}

Матрица может состоять из одной строки, из одного столбца и даже
из одного элемента. Если все элементы матрицы равны нулю, то её
называют \term{нулевой} и обозначают символом $O$.
\begin{defnt}
Если число столбцов матрицы равно числу строк: $m=n$, то матрица
называется квадратной $n$--го порядка.
\end{defnt}
Линия, вдоль которой в квадратной матрице стоят элементы с равными
номерами строк и столбцов, обычно называют главной диагональю
квадратной матрицы.
\begin{defnt}
Квадратная матрица, у которой на главной диагонали находятся
единицы, а все остальные элементы равны нулю, называется
единичной.
\end{defnt}
Т.е. единичная матрица выглядит так:
\begin{equation}\label{eq:edmatrix}
E = \begin{pmatrix}
  1 & 0 & \cdots & 0 \\
  0 & 1 & \cdots & 0 \\
  \vdots & \vdots & \ddots & \vdots \\
  0 & 0 & \cdots & 1 \\
\end{pmatrix}.
\end{equation}
В общем случае, квадратная матрица, у которой все элементы равны
нулю, за исключением элементов, стоящих на главной диагонали,
называется \term{диагональной}.

Матрица (\emph{лат.} matrix --- <<матка>>, <<источник>>, <<начало>>) --- важнейшее понятие в
современной математике. Впервые появилось в работах
\tooltip{Сильвестра}{silvestr}~и~\tooltip{Кэли}{cayley}.

Важнейшими операциями над матрицами являются:
\begin{description}
    \item[\textcolor{red}{\ding{229}}] сложение (вычитание) матриц,
    \item[\sym] умножение матрицы на число,
    \item[\sym] умножение одной матрицы на другую,
    \item[\sym] транспонирование.
\end{description}

\term{Операции сложения и вычитания матриц} выполняются
поэлементно и применимы лишь к матрицам одинаковой размерности.
\begin{defnt}
Суммой (разностью) матриц $A$ и $B$, имеющих одинаковую
размерность, является матрица $C$, той же размерности, элементы
которой есть сумма (разность) соответствующих элементов исходных
матриц: $C =A\pm B$ $\Leftrightarrow$ $c_{ij}=a_{ij}\pm b_{ij}$.
\end{defnt}

\term{Операция умножения матрицы на произвольное вещественное
число} тоже выполняется поэлементно, но применима к любым
матрицам.
\begin{defnt}
Произведением произвольной матрицы $A$ на вещественное число
$\lambda$ является матрица $C$, той же размерности, что и $A$,
элементы которой есть произведение соответствующих элементов
исходной матрицы на число $\lambda$: $C =\lambda \cdot A$
$\Leftrightarrow$ $c_{ij}=\lambda \cdot a_{ij}$.
\end{defnt}

В приведенных выше определениях использовался символ <<$\Leftrightarrow$>>, называемый символом эквивалентности.
Математическое выражение $\alpha\Leftrightarrow\beta$ читается так:
<<$\alpha$ эквивалентно $\beta$>> или <<$\alpha$ истинно
тогда и только тогда, когда истинно $\beta$>>.

Перечислим основные свойства операций сложения матриц и умножения
матриц на вещественные числа:

\begin{description}
    \item[\sym\quad \tooltip{Коммутативность:}{commut}] $A+B=B+A$.
    \item[\sym\quad \tooltip{Ассоциативность:}{assot}] $\left(A+B\right)+C=A+\left(B+C\right)$.
    \item[\sym\quad \tooltip{Дистрибутивность:}{distrib}] $\lambda\cdot\left(A+B\right)=\lambda\cdot A+\lambda\cdot B$.
\end{description}


Нулевая матрица $O$ относительно операции сложения матриц обладает
теми же свойствами, что и число 0 относительно операции сложения
вещественных чисел. Это значит, что какова бы ни была матрица $A$,
справедливы следующие равенства: $A+O=O+A=A$. В этих равенствах
размерности матриц $A$ и $O$ должны быть равны.

\begin{examp}\label{prim:deistvmatr1}
Даны матрицы:
$A=\begin{pmatrix}
      3 & -4 & 1 \\
      9 & 3 & 0 \\
    \end{pmatrix}$,
$B=\begin{pmatrix}
      4 & -9 & 3 \\
      -2 & 2 & 1 \\
    \end{pmatrix}$.

Найти матрицу $4A -3B$. \\{\bf Решение. }
\begin{displaymath}
4A=4\cdot\begin{pmatrix}
      3 & -4 & 1 \\
      9 & 3 & 0 \\
    \end{pmatrix}=\begin{pmatrix}
      12 & -16 & 4 \\
      36 & 12 & 0 \\
    \end{pmatrix}.
\end{displaymath}
\begin{displaymath}
3B=3\cdot\begin{pmatrix}
       4 & -9 & 3 \\
      -2 & 2 & 1 \\
    \end{pmatrix}=\begin{pmatrix}
       12 & -27 & 9 \\
      -6 & 6 & 3 \\
    \end{pmatrix}.
\end{displaymath}
\begin{displaymath}
4A-3B=\begin{pmatrix}
      12 & -16 & 4 \\
      36 & 12 & 0 \\
    \end{pmatrix}-\begin{pmatrix}
       12 & -27 & 9 \\
      -6 & 6 & 3 \\
    \end{pmatrix}=\begin{pmatrix}
       0 & 11 & -5 \\
      42 & 6 & -3 \\
    \end{pmatrix}.
\end{displaymath}
\end{examp}


\term{Произведение матриц} можно вычислить лишь в том случае,
когда количество столбцов в первой матрице совпадает с количеством
строк во второй.
\begin{defnt}   Произведением матрицы $A$, имеющей размерность $m\times
p$, на матрицу $B$, имеющей размерность $p\times n$, называется
матрица $C$, размерности $m\times n$: $C=A\cdot B$, если элементы
матрицы $C$ могут быть вычислены по следующей формуле:
\begin{equation}\label{eq:proizvmatrix}
    c_{ij}=\sum^p_{k=1}a_{ik}b_{kj}=a_{i1}b_{1j}+a_{i2}b_{2j}+\ldots+a_{ip}b_{pj}.
\end{equation}
\end{defnt}

Символ $\sum^p_{k=1}$, использованный в определении, означает сумму
по элементам некоторого упорядоченного множества
чисел. Что именно нужно суммировать, написано правее знака $\sum$.
При этом номера $k$ элементов, включаемых в сумму, принимают
значения $1,2,\ldots,p$.

Свойства операции умножения матриц:
\begin{description}
    \item[\sym] $A\cdot B\neq B\cdot A$ --- произведение матриц не коммутативно;
    \item[\sym] $\left(A\cdot B\right)\cdot C=A\cdot\left(B\cdot C\right)$ ---
    произведение матриц ассоциативно;
    \item[\sym] $A\cdot \left(B+C\right)=A\cdot B+A\cdot C$ ---
    операция умножения матриц дистрибутивна.
\end{description}

Единичная матрица $E$ относительно операции умножения квадратных
матриц обладает теми же свойствами, что и число 1 относительно
операции умножения вещественных чисел. Это значит, что какова бы
ни была квадратная матрица $A$, справедливы следующие равенства:
$A\cdot E=E\cdot A=A$. Правда нужно помнить, что в этих равенствах
размерности матриц $A$ и $E$ должны быть равны.


Операция \term{транспонирования} может быть применена к любым
матрицам. Она состоит в замене строк матрицы на её столбцы, т.е.
после транспонирования первый столбец исходной матрицы становится
первой строкой новой матрицы, второй столбец становится второй
строкой и т.д. Если матрица $A$ --- исходная, то транспонированная
матрица обозначается символом $A^{T}$.

\begin{examp}\label{prim:deistvmatr2}
Найти матрицы $A^{2}$, $B\cdot C$ и $B^{T}$, если
\begin{equation*}
    A=\begin{pmatrix}
      -1 & 2 \\
      3 & 4 \\
    \end{pmatrix},B=\begin{pmatrix}
      0 & 3 & -4 \\
      1 & 2 & -3 \\
    \end{pmatrix},C=\begin{pmatrix}
      -5 \\
      2 \\
     -1 \\
    \end{pmatrix}.
\end{equation*}
{\bf Решение.}
\vspace*{-0.5cm}\begin{multline*}
1)\quad A^{2}=A\cdot A=\begin{pmatrix}
      -1 & 2 \\
      3 & 4 \\
    \end{pmatrix}\cdot \begin{pmatrix}
      -1 & 2 \\
      3 & 4 \\
    \end{pmatrix}=\\ =\begin{pmatrix}
      -1\cdot(-1) +2\cdot3 & -1\cdot2 +2\cdot4 \\
      3\cdot(-1)+ 4\cdot3& 3\cdot2+ 4\cdot4 \\
    \end{pmatrix}=\begin{pmatrix}
      7 & 6 \\
      9& 22 \\
    \end{pmatrix}.
\end{multline*}\vspace*{-0.8cm}
\begin{multline*}
2)\quad B\cdot C=\begin{pmatrix}
      0 & 3 & -4 \\
      1 & 2 & -3 \\
    \end{pmatrix}\cdot\begin{pmatrix}
      -5 \\
      2 \\
     -1 \\
    \end{pmatrix}=\\ \qquad \qquad=\begin{pmatrix}
      0\cdot(-5)+3\cdot2-4\cdot(-1) \\
      1\cdot(-5)+2\cdot2-3\cdot(-1) \\
    \end{pmatrix}=\begin{pmatrix}
      10 \\
      2 \\
    \end{pmatrix}.\\
\shoveleft{3)\quad B^{T}=\begin{pmatrix}
      0 & 3 & -4 \\
      1 & 2 & -3 \\
    \end{pmatrix}^{T}=\begin{pmatrix}
      0 & 1 \\
      3 & 2 \\
      -4 & -3 \\
    \end{pmatrix}.\hfill}
\end{multline*}
\end{examp}




\section{Определители и их свойства}

\begin{defnt}
Определителем второго порядка, соответствующим квадратной матрице
$A=\begin{pmatrix}
  a_{11} & a_{12} \\
  a_{21} & a_{22} \\
\end{pmatrix}$,
называется число, определяемое по формуле: \begin{equation}\label{eq:opred2}
\det A=\begin{vmatrix}
  a_{11} & a_{12} \\
  a_{21} & a_{22} \\
\end{vmatrix}=a_{11}a_{22}-a_{12}a_{21}.
\end{equation}
\end{defnt}

Термин <<определитель>> впервые появился в начале XIX века в
работах \tooltip{Коши}{cochy}. Однако первым понятие определителя, применительно к
системам линейных алгебраических уравнений, стал использовать в 1693
году \tooltip{Лейбниц}{leibn}. Современное обозначение определителя:
таблица чисел в вертикальных прямых скобках, ввёл Кэли.

\begin{defnt}
Определителем третьего порядка, соответствующим квадратной матрице
$A=\begin{pmatrix}
  a_{11} & a_{12} & a_{13}\\
  a_{21} & a_{22} & a_{23}\\
  a_{31} & a_{32} & a_{33}\\
\end{pmatrix}$,
называется число, определяемое по формуле: \begin{multline}\label{eq:opred3}
\det A=\begin{vmatrix}
  a_{11} & a_{12} & a_{13}\\
  a_{21} & a_{22} & a_{23}\\
  a_{31} & a_{32} & a_{33}\\
\end{vmatrix}
=a_{11}a_{22}a_{33}+a_{12}a_{23}a_{31}+a_{13}a_{21}a_{32}-\\
-a_{13}a_{22}a_{31}-a_{11}a_{23}a_{32}-a_{12}a_{21}a_{33}.
\end{multline}
\end{defnt}

Формула~\refform{eq:opred3} вычисления определителя третьего порядка кажется
сложной для запоминания. Однако можно предложить правило
(так называемое \term{правило Саррюса}), с
помощью которого эту формулу можно воспроизвести без особых
усилий. Возьмем определитель третьего порядка и выпишем справа от
него еще раз первый и второй столбцы. В полученной таким образом
матрице выделим шесть диагоналей: три главных и три побочных.

\begin{minipage}[c]{0.7\textwidth}
\centering{\begin{tikzpicture}[scale=1.5]
\path   (0,2) node (a11) {$a_{11}$}
        (1,1) node (a22) {$a_{22}$}
        (2,0) node (a33) {$a_{33}$}
        (1,2) node (a12) {$a_{12}$}
        (2,2) node (a13) {$a_{13}$}
        (0,1) node (a21) {$a_{21}$}
        (2,1) node (a23) {$a_{23}$}
        (0,0) node (a31) {$a_{31}$}
        (1,0) node (a32) {$a_{32}$}
        (3,2) node (a11d) {$a_{11}$}
        (3,1) node (a21d) {$a_{21}$}
        (3,0) node (a31d) {$a_{31}$}
        (4,2) node (a12d) {$a_{12}$}
        (4,1) node (a22d) {$a_{22}$}
        (4,0) node (a32d) {$a_{32}$};
    \draw (-0.3,-0.2) -- (-0.3,2.2);
    \draw (2.3,-0.2) -- (2.3,2.2);
    \draw[draw=p!50,double=blue,very thick]  (a13)--(a22)--(a31);
    \draw[draw=p!50,double=blue,very thick]  (a11d)--(a23)--(a32);
    \draw[draw=p!50,double=blue,very thick]  (a12d)--(a21d)--(a33);
    \draw[draw=p!50,double=red,very thick]  (a11)--(a22)--(a33);
    \draw[draw=p!50,double=red,very thick]  (a12)--(a23)--(a31d);
    \draw[draw=p!50,double=red,very thick]  (a13)--(a21d)--(a32d);
    \end{tikzpicture}}
\end{minipage}
Тройки чисел, образующие главные диагонали  и отмеченные на рисунке
красными линиями, входят в формулу для определителя третьего
порядка со знаком <<плюс>>. Тройки чисел, образующие побочные
диагонали (отмечены на рисунке синими линиями), входят  в эту
формулу со знаком <<минус>>.

Свойства определителей рассмотрим без доказательств
(доказательства могут быть найдены в~\cite{Ilin:2002}).
\begin{description}
    \item[Свойство 1.] Величина определителя не изменится, если
    все его строки заменить на соответствующие столбцы. Это
    значит, что для произвольной квадратной матрицы $A$: $\det A = \det
    A^T$.
    \item[Свойство 2.] При перестановке любых двух строк (или двух
    столбцов) определителя, его знак меняется на противоположный.
    \item[Свойство 3.] Если элементы двух строк (или двух
    столбцов) определителя пропорциональны (в частности, равны),
    то такой определитель равен нулю.
    \item[Свойство 4.] Если все элементы некоторой строки (или
    столбца) определителя равны нулю, то и сам определитель равен
    нулю.
    \item[Свойство 5.] Общий множитель всех элементов некоторой
    строки (или столбца) определителя можно вынести за знак этого
    определителя.
    \item[Свойство 6.] Если к элементам некоторой строки (или
    столбца) прибавить соответствующие элементы
    другой строки (столбца), предварительно умноженные на
    произвольное число, не равное нулю, то величина определителя
    не изменится.
\end{description}
Разумеется все перечисленные свойства справедливы для
определителей любого порядка.

Существует еще одно свойство определителей, но для его
формулировки нам потребуются два новых понятия.

Вернемся к формуле~\refform{eq:opred3}:
\begin{multline*}\begin{vmatrix}
  a_{11} & a_{12} & a_{13}\\
  a_{21} & a_{22} & a_{23}\\
  a_{31} & a_{32} & a_{33}\\
\end{vmatrix}=a_{11}a_{22}a_{33}+a_{12}a_{23}a_{31}+a_{13}a_{21}a_{32}-\\
-a_{13}a_{22}a_{31}-a_{11}a_{23}a_{32}-a_{12}a_{21}a_{33}.
\end{multline*}
Сформируем в правой части этого равенства три группы из слагаемых, содержащих элементы  первой
строки определителя: $a_{11}$, $a_{12}$ и $a_{13}$,  и вынесем в
каждой группе соответствующий элемент за скобку. В результате
получим:

\begin{multline*}
\begin{vmatrix}
  a_{11} & a_{12} & a_{13}\\
  a_{21} & a_{22} & a_{23}\\
  a_{31} & a_{32} & a_{33}\\
\end{vmatrix}=a_{11}\left(a_{22}a_{33}-a_{23}a_{32}\right)+a_{12}\left(a_{23}a_{31}-a_{21}a_{33}\right)+\\
+a_{13}\left(a_{21}a_{32}-a_{22}a_{31}\right).
\end{multline*}
Выражение, стоящее в скобках после элемента $a_{11}$, называется
его \term{алгебраическим дополнением} и обозначается символом
$A_{11}$, т.\,е. $A_{11}=a_{22}a_{33}-a_{23}a_{32}$. Аналогично,
$A_{12}=a_{23}a_{31}-a_{21}a_{33}$ --- алгебраическое дополнение
элемента $a_{12}$, $A_{13}=a_{21}a_{32}-a_{22}a_{31}$
--- алгебраическое дополнение элемента $a_{13}$. С учетом этих
новых обозначений формула для вычисления определителя третьего
порядка принимает вид:
\begin{equation}\label{eq:str1opred3}
\begin{vmatrix}
  a_{11} & a_{12} & a_{13}\\
  a_{21} & a_{22} & a_{23}\\
  a_{31} & a_{32} & a_{33}\\
\end{vmatrix}=a_{11}A_{11}+a_{12}A_{12}+a_{13}A_{13}.
\end{equation}
Формула~\refform{eq:str1opred3} называется разложением определителя по элементам
первой строки. Аналогичные разложения можно записать для элементов
любой строки и любого столбца определителя.

\begin{defnt}
Минором $M_{ij}$ элемента $a_{ij}$ определителя $n$-го порядка,
называется определитель $\left(n-1\right)$-го порядка, получаемый
из данного определителя вычёркиванием $i$-й строки и $j$-го
столбца, т.\,е. той строки и того столбца, на пересечении которых
стоит элемент $a_{ij}$.
\end{defnt}

\begin{examp}\label{prim:minor}
\centering{\begin{tikzpicture}[scale=1,transform shape]
\path   (0,2) node (a11) {$a_{11}$}
        (1,1) node (a22) {$a_{22}$}
        (2,0) node (a33) {$a_{33}$}
        (1,2) node (a12) {$a_{12}$}
        (2,2) node (a13) {$a_{13}$}
        (0,1) node (a21) {$a_{21}$}
        (2,1) node (a23) {$a_{23}$}
        (0,0) node (a31) {$a_{31}$}
        (1,0) node (a32) {$a_{32}$}
        (3,1) node (sled) {$\Rightarrow$}
        (5,1) node (minor)[fill=green!50,draw] {$M_{11}=\begin{vmatrix}
          a_{22} & a_{23} \\
          a_{32} & a_{33} \\
        \end{vmatrix}$}
        (8.2,0.7) node (comm)[text width=3cm, text badly centered] {--- минор элемента $a_{11}$}
        (0,-1) node (a11d) {$a_{11}$}
        (1,-2) node (a22d) {$a_{22}$}
        (2,-3) node (a33d) {$a_{33}$}
        (1,-1) node (a12d) {$a_{12}$}
        (2,-1) node (a13d) {$a_{13}$}
        (0,-2) node (a21d) {$a_{21}$}
        (2,-2) node (a23d) {$a_{23}$}
        (0,-3) node (a31d) {$a_{31}$}
        (1,-3) node (a32d) {$a_{32}$}
        (3,-2) node (sledd) {$\Rightarrow$}
        (5,-2) node (minord)[fill=green!50,draw] {$M_{32}=\begin{vmatrix}
          a_{11} & a_{13} \\
          a_{21} & a_{23} \\
        \end{vmatrix}$}
        (8.2,-2.3) node (commd)[text width=3cm, text badly centered] {--- минор элемента $a_{32}$};
    \draw (-0.4,-0.2) -- (-0.4,2.2);
    \draw (2.4,-0.2) -- (2.4,2.2);
    \draw (-0.4,-3.2) -- (-0.4,-0.8);
    \draw (2.4,-3.2) -- (2.4,-0.8);
    \draw[draw=pex,double=red, very thin]  (-0.2,-0.2)--(-0.2,2.2);
    \draw[draw=pex,double=red, very thin]  (-0.3,2.03)--(2.3,2.03);
    \draw[draw=pex,double=red, very thin]  (0.8,-3.2)--(0.8,-0.8);
    \draw[draw=pex,double=red, very thin]  (-0.3,-2.97)--(2.3,-2.97);
\end{tikzpicture}}
\end{examp}

Минор и алгебраическое дополнение элемента определителя связаны
друг с другом простым соотношением. Нетрудно убедиться в
справедливости следующего правила:
\begin{defnt}
Алгебраическое дополнение $A_{ij}$ элемента $a_{ij}$ определителя
и минор $M_{ij}$ этого элемента связаны равенством
\begin{equation}\label{eq:algdop}
    A_{ij}=\left(-1\right)^{i+j}M_{ij}.
\end{equation}
\end{defnt}
Данное равенство можно рассматривать как определение
алгебраического дополнения.

Теперь мы готовы сформулировать последнее свойство определителей.
\begin{description}
    \item[Свойство 7.]\hfill
        \begin{itemize}
        \item Сумма произведений элементов какой-либо строки
        (столбца) определителя на соответствующие алгебраические
        дополнения элементов этой строки (столбца) равна величине
        этого определителя.
        \item Сумма произведений элементов какой-либо строки
        (столбца) определителя на соответствующие алгебраические
        дополнения элементов другой строки (столбца) равна нулю.
    \end{itemize}
\end{description}



\section{Системы линейных алгебраических уравнений и их решение методом Крамера}

\begin{defnt}
Будем называть системой $m$ линейных алгебраических уравнений с $n$ неизвестными совокупность уравнений
следующего вида:
\begin{equation}\label{eq:system}
\begin{cases}
  a_{11}x_1 + a_{12}x_2 + \cdots + a_{1n}x_n = b_1 \\
  a_{21}x_1 + a_{22}x_2 + \cdots + a_{2n}x_n = b_2 \\
  \dots\dots\dots\dots\dots\dots\dots\dots\dots\dots\\
  a_{m1}x_1 + a_{m2}x_2 + \cdots + a_{mn}x_n = b_m
\end{cases}
\end{equation}
\end{defnt}

В формуле~\refform{eq:system} $x_1,x_2,\dots,x_n$ --- неизвестные,
подлежащие определению. Коэффициенты при неизвестных образуют
матрицу:
\begin{equation*}\label{eq:systmatrix}
A = \begin{pmatrix}
  a_{11} & a_{12} & \cdots & a_{1n} \\
  a_{21} & a_{22} & \cdots & a_{2n} \\
  \vdots & \vdots & \ddots & \vdots \\
  a_{m1} & a_{m2} & \cdots & a_{mn} \\
\end{pmatrix},
\end{equation*}
называемую \term{основной матрицей системы}. Правые части уравнений системы~\refform{eq:system},
а также неизвестные, образуют два вектор-столбца:
\begin{equation*}\label{eq:stolbsvobchl}
B = \begin{pmatrix}
  b_1 \\
  b_2 \\
  \vdots \\
  b_m \\
\end{pmatrix},
X = \begin{pmatrix}
  x_1 \\
  x_2 \\
  \vdots \\
  x_n \\
\end{pmatrix},
\end{equation*}
которые называются \term{век\-то\-ром-столб\-цом сво\-бод\-ных чле\-нов} и \term{век\-то\-ром-столб\-цом не\-из\-вест\-ных}, соответственно.

Используя эти обозначения и определение произведения матриц \refform{eq:proizvmatrix}, можно
записать систему уравнений~\refform{eq:system} в матричной форме:
\begin{equation*}\label{eq:systmatrform}
A\cdot X = B.
\end{equation*}

Далее приведена диаграмма, демонстрирующая классификацию систем линейных алгебраических уравнений (СЛАУ).
На этой диаграмме используются следующие обозначения:
\begin{description}
    \item[\sym] $X\exists$ --- <<$X$ существует>>;
    \item[\sym] $X\nexists$ --- <<$X$ не существует>>;
    \item[\sym] $X\exists !$ --- <<$X$ существует и единственен>>;
    \item[\sym] $X\exists\infty$ --- <<существует бесконечно много $X$>>.
\end{description}

\begin{tikzpicture}[auto]
\tikzstyle{decision} = [aspect=2, diamond, draw=blue, thick, fill=blue!20,
text width=6em, text badly centered, inner sep=1pt];
\tikzstyle{block} = [rectangle, draw=blue, thick, fill=blue!20,
text width=9em, text centered, rounded corners, minimum height=2em];
\tikzstyle{line} = [draw, thick, -latex',shorten >=2pt];
\tikzstyle{cloud} = [draw=red, thick, ellipse, fill=red!20, minimum height=2em];
\matrix [column sep=4mm, row sep=6mm]
{
% row 1
\node [cloud] (n1) {$B=O$}; & \node [decision] (init) {СЛАУ}; & \node [cloud] (n2) {$B\neq O$}; \\
% row 2
\node [block] (one) {Однородные};& & \node [block] (two) {Неоднородные}; \\
% row 3
& \node [cloud] (n4) {$X\exists$}; & \node [cloud] (n3) {$X\nexists$};  \\
% row 4
& \node [block] (three) {Совместные}; & \node [block] (four) {Несовместные}; \\
% row 5
\node [cloud] (n5) {$X\exists !$}; & \node [cloud] (n6) {$X\exists\infty$}; &  \\
% row 6
\node [block] (five) {Определенные};& \node [block] (six) {Неопределенные}; & \\
};
\tikzstyle{every path}=[line]
\path (init) --  (n1);
\path (init) --  (n2);
\path (n1) -- (one);
\path (n2) -- (two);
\path (two) -- (n4);
\path (two) -- (n3);
\path (one) -- (n4);
\path (n3) -- (four);
\path (n4) -- (three);
\path (three) -- (n5);
\path (three) -- (n6);
\path (n5) -- (five);
\path (n6) -- (six);
\end{tikzpicture}

\begin{defnt}
Решить систему линейных алгебраических уравнений --- это значит найти такие значения неизвестных
$x_1=x_1^{*},x_2=x_2^{*},\dots,x_n=x_n^{*}$, которые после подстановки в~\refform{eq:system}
обращают каждое уравнение этой системы в тождество.
\end{defnt}

В случае, если $m=n$, т.\,е. число уравнений равно числу неизвестных, и $\det A \neq 0$
решение системы линейных алгебраических уравнений \refform{eq:system} единственно и может быть найдено методом \tooltip{Крамера}{kramer}.
Рассмотрим этот метод на примере системы из трёх уравнений на три неизвестных:
\begin{equation*}\label{eq:triuravn}
\begin{cases}
  a_{11}x_1 + a_{12}x_2 + a_{13}x_3 = b_1 \\
  a_{21}x_1 + a_{22}x_2 + a_{23}x_3 = b_2 \\
  a_{31}x_1 + a_{32}x_2 + a_{33}x_3 = b_3
\end{cases}
\Leftrightarrow
\begin{pmatrix}
  a_{11} & a_{12} & a_{13} \\
  a_{21} & a_{22} & a_{23} \\
  a_{31} & a_{32} & a_{33} \\
\end{pmatrix}\cdot
\begin{pmatrix}
  x_1 \\
  x_2 \\
  x_3 \\
\end{pmatrix}=
\begin{pmatrix}
  b_1 \\
  b_2 \\
  b_3 \\
\end{pmatrix}.
\end{equation*}

В методе Крамера необходимо вычислить несколько определителей:
\begin{enumerate}
  \item Главный определитель системы:
$\Delta=\det A=\begin{vmatrix}
  a_{11} & a_{12} & a_{13}\\
  a_{21} & a_{22} & a_{23}\\
  a_{31} & a_{32} & a_{33}\\
\end{vmatrix}.$
  \item Определители неизвестных $x_1,x_2,x_3$:
 \begin{equation*}\Delta_1=\begin{vmatrix}
  b_1 & a_{12} & a_{13}\\
  b_2 & a_{22} & a_{23}\\
  b_3 & a_{32} & a_{33}\\
\end{vmatrix}, \Delta_2=\begin{vmatrix}
  a_{11} & b_1 & a_{13}\\
  a_{21} & b_2 & a_{23}\\
  a_{31} & b_3 & a_{33}\\
\end{vmatrix}, \Delta_3=\begin{vmatrix}
  a_{11} & a_{12} & b_1\\
  a_{21} & a_{22} & b_2\\
  a_{31} & a_{32} & b_3\\
\end{vmatrix}.\end{equation*}
\end{enumerate}
После этого решение СЛАУ может быть найдено по формулам Крамера:
\begin{equation}\label{eq:kramer}
x_1=\frac{\Delta_1}{\Delta},x_2=\frac{\Delta_2}{\Delta},x_3=\frac{\Delta_3}{\Delta}.
\end{equation}

\begin{examp}\label{prim:kramer}
Решить методом Крамера:
$\begin{cases}
  -x_1 + 2x_2 + 3x_3 = -1 \\
  2x_1 -5x_2 + x_3 = 10 \\
  4x_1 + 3x_2 -7x_3 =-2
\end{cases}.$
\\{\bf Решение.}
\begin{multline*}\Delta=\begin{vmatrix}
  -1 & 2 & 3\\
  2 & -5 & 1\\
  4 & 3 & -7\\
\end{vmatrix}=-1\cdot(-1)^{2}\begin{vmatrix}
  -5 & 1\\
  3 & -7\\
\end{vmatrix}+ 2\cdot(-1)^{3}\begin{vmatrix}
  2 &  1\\
  4 & -7\\
\end{vmatrix}+ \\ +3\cdot(-1)^{4}\begin{vmatrix}
  2 & -5 \\
  4 & 3 \\
\end{vmatrix}= -32+36+78=82.
\end{multline*}
\begin{multline*}\Delta_1=\begin{vmatrix}
  -1 & 2 & 3\\
  10 & -5 & 1\\
  -2 & 3 & -7\\
\end{vmatrix}=-1\cdot(-1)^{2}\begin{vmatrix}
  -5 & 1\\
  3 & -7\\
\end{vmatrix}+ 2\cdot(-1)^{3}\begin{vmatrix}
  10 &  1\\
  -2 & -7\\
\end{vmatrix}+ \\ +3\cdot(-1)^{4}\begin{vmatrix}
  10 & -5 \\
  -2 & 3 \\
\end{vmatrix}= -32+136+60=164.
\end{multline*}
Аналогично: $\Delta_2=\begin{vmatrix}
  -1 & -1 & 3\\
  2 & 10 & 1\\
  4 & -2 & -7\\
\end{vmatrix}=-82,
\Delta_3=\begin{vmatrix}
  -1 & 2 & -1\\
  2 & -5 & 10\\
  4 & 3 & -2\\
\end{vmatrix}=82.$
Отсюда по формулам Крамера \refform{eq:kramer} находим:
\begin{equation*}
    x_1=\frac{164}{82}=2,
    x_2=\frac{-82}{82}=-1,
    x_3=\frac{82}{82}=1.
\end{equation*}
\textbf{Ответ:} $(2;-1;1)$.
\end{examp}

\section{Векторная алгебра}
\subsection{Векторы и линейные операции над ними}
Вектором называется направленный отрезок прямой.

Для обозначения векторов используются символы $\vec{a}$ или $\overrightarrow{AB}$, где точка $A$ --- начало вектора, а точка $B$ --- конец вектора.

Длиной вектора (обозначается $\left|\vec{a}\right|$ или $\left|\overrightarrow{AB}\right|$) называется расстояние между его началом и концом. Нулевым называется вектор, длина которого равна нулю, а направление не определено.

Векторы равны, если равны их длины, а направления совпадают.

Коллинеарными называют векторы, расположенные на параллельных прямых (в частности, на одной прямой), а компланарными "--- векторы, расположенные в параллельных плоскостях (в частности, в одной плоскости).

\begin{figure}[H]
	\begin{floatrow}
		\ffigbox
		{\begin{tikzpicture}[>=latex,thick]
			\draw[->] (0,0) -- node[above,sloped] {$\vec{a}$} (2,2);
			\draw[->] (2,2) -- node[above,sloped] {$\vec{b}$} (4,0);
			\draw[->] (0,0) -- node[below,sloped] {$\vec{a}+\vec{b}$} (4,0);
			\end{tikzpicture}}
		{\caption{Определение суммы векторов с помощью правила треугольника}\label{sum1}}
		\ffigbox[\Xhsize]
		{\begin{tikzpicture}[>=latex,thick]
			\draw[->] (0,0) -- node[above,sloped] {$\vec{a}$} (1,2);
			\draw[->] (0,0) -- node[below,sloped] {$\vec{b}$} (4,0);
			\draw[->] (0,0) -- node[above,sloped] {$\vec{a}+\vec{b}$} (5,2);
			\draw[dashed] (1,2) -- (5,2);
			\draw[dashed] (4,0) -- (5,2);
			\end{tikzpicture}}
		{\caption{Определение суммы векторов с помощью правила параллелограмма}\label{sum2}}
	\end{floatrow}
\end{figure}


Суммой $\vec{a}+\vec{b}$ векторов $\vec{a}$ и $\vec{b}$, при условии, что конец вектора $\vec{a}$ совмещен с началом $\vec{b}$, называется вектор $\vec{c}$, соединяющий начало вектора $\vec{a}$ с концом $\vec{b}$. Это "--- так называемое правило треугольника (рис.~\ref{sum1}). Кроме него, для вычисления суммы векторов, есть ещё правило параллелограмма, применение которого проиллюстрировано на рисунке~(рис.~\ref{sum2}).

Разностью векторов $\vec{a}$ и $\vec{b}$ называется вектор $\vec{c}=\vec{a}-\vec{b}$, для которого $\vec{c}+\vec{b}=\vec{a}$  (см. рис.~\ref{razn}).

\begin{figure}[H]
	\ffigbox
	{\begin{tikzpicture}[>=latex,thick]
		\draw[->] (0,0) -- node[above,sloped] {$\vec{b}$} (2,2);
		\draw[->] (0,0) -- node[below,sloped] {$\vec{a}$} (5,0);
		\draw[->] (2,2) -- node[above,sloped] {$\vec{a}-\vec{b}$} (5,0);
		\end{tikzpicture}}
	{\caption{Определение разности векторов}\label{razn}}
\end{figure}

Произведением $\lambda\vec{a}$ вектора $\vec{a}$ на число $\lambda$ называется вектор $\vec{b}$ такой, что:
\begin{enumerate}
	\item $\left|\vec{b}\right|=\left|\lambda\right|\left|\vec{a}\right|$;
	\item $\vec{b}$ коллинеарен вектору $\vec{a}$  и направлен в ту же сторону при $\lambda>0$ ($\vec{a\upuparrows\vec{b}}$) и в противоположную сторону "--- при $\lambda<0$ ($\vec{a}\uparrow\/\downarrow\vec{b}$).
\end{enumerate}

Линейной комбинацией векторов  $\vec{a}_1,\vec{a}_2,\ldots,\vec{a}_n$, называется вектор
\begin{equation}\label{lincomb}
\vec{c}=\alpha_1\vec{a}_1+\alpha_2\vec{a}_2+\ldots+\alpha_n\vec{a}_n=\sum_{k=1}^n\alpha_k\vec{a}_k,
\end{equation}
где $\alpha_1,\alpha_2,\ldots,\alpha_n$ "--- произвольные действительные числа.

Векторы $\vec{a}_1,\vec{a}_2,\ldots,\vec{a}_n$ "--- линейно зависимы, если  существуют такие числа  $\alpha_1,\alpha_2,\ldots,\alpha_n$, для которых $\alpha_1^2+\alpha_2^2+\ldots+\alpha_n^2\neq 0$, а линейная комбинация~(\ref{lincomb}) равна нулю. В противном случае эти векторы называются линейно независимыми.

Базисом на плоскости и в пространстве называется максимальная линейно независимая на плоскости или в пространстве система векторов (добавление к системе еще одного вектора делает ее линейно зависимой).

Пусть   $\vec{e}_1,\vec{e}_2,\vec{e}_3$ "--- базис в пространстве, тогда любой вектор $\vec{a}$  пространства разлагается единственным образом по базисным векторам:
\[
\vec{a}=\alpha_1\vec{e}_1+\alpha_2\vec{e}_2+\alpha_3\vec{e}_3.
\]
Коэффициенты этого разложения называются координатами вектора $\vec{a}$  в базисе  $\vec{e}_1,\vec{e}_2,\vec{e}_3$:
$\vec{a}=\left\{\alpha_1;\alpha_2;\alpha_3\right\}$.

Пусть $\vec{a}=\left\{\alpha_1;\alpha_2;\alpha_3\right\}$, $\vec{b}=\left\{\beta_1;\beta_2;\beta_3\right\}$. Линейные операции над векторами в координатном представлении имеют вид:
\begin{enumerate}
	\item $\vec{a}\pm\vec{b}=\left\{\alpha_1\pm\beta_1;\alpha_2\pm\beta_2;\alpha_3\pm\beta_3\right\}$;
	\item $\lambda\vec{a}=\left\{\lambda\alpha_1;\lambda\alpha_2;\lambda\alpha_3\right\}$.
\end{enumerate}
Если векторы $\vec{a}$ и $\vec{b}$ равны, то $\alpha_1=\beta_1$, $\alpha_2=\beta_2$, $\alpha_3=\beta_3$.

\subsection{Проекция вектора на ось}
Осью называется направленная прямая.

Предположим, в пространстве заданы вектор $\overrightarrow{AB}$ и ось $l$.
Пусть $A_l$ --- проекция точки $A$ на ось $l$, т.е. основание перпендикуляра, опущенного из точки $A$ на эту ось, $B_l$ --- проекция точки $B$ на ось $l$. Проекцией вектора $\overrightarrow{AB}$ на ось $l$ называется модуль вектора $\overrightarrow{A_lB_l}$, взятый со знаком <<$+$>>, если вектор $\overrightarrow{A_lB_l}$ направлен также, как и ось $l$, и --- со знаком <<$-$>>, если вектор $\overrightarrow{A_lB_l}$ направлен противоположно оси $l$. Проекция вектора $\overrightarrow{AB}$ на ось $l$ обозначается через $Пр_l\overrightarrow{AB}$ и вычисляется по формуле:
\begin{equation}\label{proek}
Пр_l\overrightarrow{AB}=\left|\overrightarrow{AB}\right|\cos \varphi,
\end{equation}
где $\varphi$ --- угол между вектором $\overrightarrow{AB}$ и осью $l$ (см. рис.~\ref{pr}).

\begin{figure}[H]
	\ffigbox
	{\begin{tikzpicture}[>=latex,thick]
		\draw[->] (1.5,1) -- (4,2);
		\draw[dashed] (-1,0) -- (1.5,1);
		\draw[->] (-1.5,0) --  (5.5,0);
		\draw (5.5,0) node[anchor=south] {$l$};
		\draw[dashed] (1.5,1) -- (1.5,0);
		\draw[dashed] (4,2) -- (4,0);
		\fill (1.5,0) circle(1.5pt);
		\fill (4,0) circle(1.5pt);
		\draw (1.5,0) node[anchor=north] {$A_l$};
		\draw (4,0) node[anchor=north] {$B_l$};
		\draw (1.5,1) node[anchor=south] {$A$};
		\draw (4,2) node[anchor=south] {$B$};
		\draw (0,0) to [controls=+(22:0.1) and +(-22:0.1)] node[right] {$\varphi$} (-0.07,0.37);
		\draw (1.5,0.2) --  (1.7,0.2) -- (1.7,0);
		\draw (4,0.2) --  (4.2,0.2) -- (4.2,0);
		\end{tikzpicture}}
	{\caption{Проекция вектора на ось}\label{pr}}
\end{figure}

Проекция вектора на ось обладает очевидными свойствами:
\begin{enumerate}
	\item $Пр_l\left(\vec{a}+\vec{b}\right)=Пр_l\vec{a}+Пр_l\vec{b}$;
	\item $Пр_l\left(k\vec{a}\right)=kПр_l\vec{a}$.
\end{enumerate}

\subsection{Прямоугольная система координат}
Прямоугольной (декартовой) системой координат называется совокупность точки $O$ и ортонормированного базиса $\vec{i},\vec{j},\vec{k}$, т.е. такого базиса, в котором векторы имеют длины, равные $1$, и взаимно перпендикулярны. Три взаимно перпендикулярные прямые в направлении базисных векторов $\vec{i},\vec{j},\vec{k}$ называются осями координат: осью абсцисс, ординат и аппликат, соответственно.

Произвольный вектор $\vec{a}$ пространства разлагается единственным образом по базисным векторам $\vec{i},\vec{j},\vec{k}$:
$\vec{a}=a_x\vec{i}+a_y\vec{j}+a_z\vec{k}$. Длина вектора $\vec{a}$ в прямоугольной системе координат равна:
\begin{equation}\label{eq:dlin_vec}
\left|\vec{a}\right|=\sqrt{a_x^2+a_y^2+a_z^2\mathstrut}.
\end{equation}

Декартовы координаты вектора $\vec{a}=\left\{a_x;a_y;a_z\right\}$  совпадают с его проекциями на соответствующие оси координат:
\[
a_x=Пр_x\vec{a},\quad a_y=Пр_y\vec{a},\quad a_z=Пр_z\vec{a}.
\]

Проекции вектора на оси координат могут быть вычислены по формулам:
\[
Пр_x\vec{a}=\left|\vec{a}\right|\cos\alpha,\quad Пр_y\vec{a}=\left|\vec{a}\right|\cos\beta,\quad Пр_z\vec{a}=\left|\vec{a}\right|\cos\gamma,
\]
где $\alpha,\beta,\gamma$ "--- углы наклона вектора $\vec{a}$ к соответствующим осям координат. Косинусы этих углов: $\cos\alpha,\cos\beta,\cos\gamma$, называемые направляющими косинусами, выражаются через декартовы координаты с помощью формул:
\begin{equation}\label{eq:napr}
\cos\alpha=\frac{a_x}{\left|\vec{a}\right|},\quad \cos\beta=\frac{a_y}{\left|\vec{a}\right|},\quad \cos\gamma=\frac{a_z}{\left|\vec{a}\right|}.
\end{equation}
\[
\cos^2\alpha+\cos^2\beta+\cos^2\gamma=1.
\]

\begin{examp}
	Найти длину вектора $\vec{a}=\left\{1;2;-3\right\}$ и его направляющие косинусы.
	\par\textbf{Решение.}
	$\left|\vec{a}\right|=\sqrt{1^2+2^2+(-3)^2\mathstrut}=\sqrt{14}$. Отсюда с помощью формул~(\ref{eq:napr}) находим:
	\[
	\cos\alpha=\frac{1}{\sqrt{14}},\quad \cos\beta=\frac{2}{\sqrt{14}},\quad \cos\gamma=\frac{-3}{\sqrt{14}}.
	\]
\end{examp}

Пусть заданы векторы $\vec{a}=\left\{a_x;a_y;a_z\right\}$ и $\vec{b}=\left\{b_x;b_y;b_z\right\}$. Если эти векторы коллинеарны, то, в соответствии с определением произведения вектора на число, они отличаются друг от друга числовым множителем, т.е. $\vec{b}=\lambda\vec{a}$. Откуда получаем:
\[
\left\{b_x;b_y;b_z\right\}=\lambda\left\{a_x;a_y;a_z\right\}\quad\Leftrightarrow\quad
\frac{b_x}{a_x}=\frac{b_y}{a_y}=\frac{b_z}{a_z}.
\]

\begin{examp}
	Определить, при каких значениях $\alpha$ и $\beta$ векторы $\vec{a}=\left\{2;\alpha;1\right\}$ и  $\vec{b}=\left\{3;-6;\beta\right\}$ коллинеарны.
	\par\textbf{Решение.}
	Координаты данных векторов пропорциональны: $\frac{3}{2}=\frac{-6}{\alpha}=\frac{\beta}{1}$. Отсюда находим, что $\alpha=-4$, $\beta=\tfrac{3}{2}$. При этих значениях $\alpha$ и $\beta$ векторы коллинеарны.
\end{examp}

Если заданы точки $A\left(x_1,y_1,z_1\right)$ и $B\left(x_2,y_2,z_2\right)$, то декартовы координаты вектора $\overrightarrow{AB}$  равны:
\begin{equation}\label{conec_nachalo}
\overrightarrow{AB} =\left\{x_2-x_1,y_2-y_1,z_2-z_1\right\}.
\end{equation}

\subsection{Цилиндрическая и сферическая системы координат}
Часто бывает удобно задавать положение точек в пространстве с помощью не прямоугольных, а криволинейных координат. Наиболее употребительными из этих систем координат являются цилиндрические и сферические координаты.

\begin{figure}[H]
	\ffigbox
	{\begin{tikzpicture}[thick,>=stealth']
		\draw[dashed] (0,0) to [controls=+(90:1) and +(90:1)] (3,0);
		\draw (0,0) to [controls=+(-90:1) and +(-90:1)] (3,0);
		\draw (0,4) to [controls=+(90:1) and +(90:1)] (3,4);
		\draw (0,4) to [controls=+(-90:1) and +(-90:1)] (3,4);
		\draw (0,0) --  (0,4);
		\draw (3,0) -- (3,4);
		\draw[->] (3,0) -- (4.2,0);
		\draw[dashed] (1.5,0) -- (3,0);
		\draw[->] (1.5,4) -- (1.5,5.2);
		\draw[dashed] (1.5,0) -- node[left] {$z$} (1.5,4);
		\draw[->] (0.5,-0.58) -- (-0.5,-1.16);
		\draw[dashed] (1.5,0) -- (0.5,-0.58);
		\draw (1.5,4) -- (0.5,3.42);
		\draw (0.5,3.42) --  (0.5,-0.58);
		\draw[dashed] (1.5,0) -- (2.5,-0.58);
		\draw (1.5,4) -- (2.5,3.42);
		\draw (2.5,3.42) -- node[left] {$z$} (2.5,-0.58);
		\draw (1.5,5.2) node[anchor=east] {Z};
		\draw (4.2,0) node[anchor=south] {Y};
		\draw (1.5,0) node[anchor=south east] {O};
		\draw (-0.5,-1.16) node[anchor=south east] {X};
		\draw (2.4,3.47) node[anchor=north west] {M};
		\draw (2.3,-0.3) node[rotate=-30] {$r$};
		\draw[->] (1.2,-0.174) to [controls=+(-45:0.22) and +(-135:0.22)] node[below,sloped] {$\varphi$} (1.8,-0.174);
		\fill (1.5,0) circle(1.5pt);
		\fill (2.5,3.42) circle(1.5pt);
		\end{tikzpicture}}
	{\caption{Цилиндрическая система координат}\label{cilindr}}
\end{figure}

Цилиндрическими координатами точки M называется тройка чисел $\left(r;\varphi;z\right)$ (см. рис.~\ref{cilindr}), где координата $r$ есть расстояние от точки М до оси OZ; $\varphi$ "--- угол, образованный плоскостью, проходящей через ось OZ и точку М, с плоскостью ХOZ; $z$ "--- обычная аппликата точки M. При этом $\varphi$ может меняться от $0$ до $2\pi$, а $r$ "--- от $0$ до $+\infty$. Прямоугольные координаты точки M связаны с цилиндрическими с помощью соотношений:
\[
x=r\cos\varphi,\quad y=r\sin\varphi,\quad z=z.
\]


\begin{figure}[H]
	\ffigbox
	{\begin{tikzpicture}[thick,>=stealth']
		\draw (1.5,2.5) to [controls=+(0:1.3) and +(90:1.3)] (4,0);
		\draw (1.5,2.5) to [controls=+(200:1.4) and +(90:1.4)] (-0.3,-1.044);
		\draw (-0.3,-1.044) to [controls=+(-18:1.3) and +(-114:1.3)] (4,0);
		\draw (1.5,2.5) to [controls=+(-25:1.3) and +(90:1.3)] (3,-0.87);
		\draw[->] (4,0) -- (4.6,0);
		\draw[dashed] (1.5,0) -- (4,0);
		\draw[->] (1.5,2.5) -- (1.5,3.1);
		\draw[dashed] (1.5,0) -- (1.5,2.5);
		\draw[->] (-0.3,-1.044) -- (-0.8,-1.334);
		\draw[dashed] (1.5,0) -- (-0.3,-1.044);
		\draw[dashed] (1.5,0) -- (2.5,-0.58);
		\draw[dashed] (1.5,2.08) -- (2.5,1.5);
		\draw[dashed] (2.5,1.5) -- (2.5,-0.58);
		\draw[dashed] (1.5,0) -- node[below,sloped] {$r$} (2.5,1.5);
		\draw (1.5,3.1) node[anchor=east] {Z};
		\draw (4.6,0) node[anchor=south] {Y};
		\draw (1.5,0) node[anchor=south east] {O};
		\draw (-0.8,-1.334) node[anchor=south east] {X};
		\draw (2.4,1.51) node[anchor=south west] {M};
		\draw[->] (1.2,-0.174) to [controls=+(-45:0.22) and +(-135:0.22)] node[below,sloped] {$\varphi$} (1.8,-0.174);
		\draw[->] (1.5,1) to [controls=+(20:0.22) and +(125:0.22)] node[above] {$\theta$} (2,0.75);
		\fill (1.5,0) circle(1.5pt);
		\fill (2.5,1.5) circle(1.5pt);
		\end{tikzpicture}}
	{\caption{Сферическая система координат}\label{sphera}}
\end{figure}

Сферическими  координатами точки М называется тройка чисел $\left(r;\varphi;\theta\right)$ (см. рис.~\ref{sphera}), где координата $r$ есть расстояние от точки М до начала координат O; $\varphi$ "--- угол, который полуплоскость, проходящая через ось OZ и точку М, образует с плоскостью ХOZ; $\theta$ "--- угол, который отрезок ОМ образует с положительным направлением оси OZ. При этом, $r$ может изменяться от $0$ до $+\infty$, угол  $\varphi$  отсчитывается против часовой стрелки от положительного направления оси ОХ и может меняться от $0$ до $2\pi$, угол $\theta$ отсчитывается от положительного направления оси OZ и может изменяться $0$ до $\pi$. Прямоугольные координаты точки M связаны со сферическими с помощью соотношений:
\[
x=r\sin\theta\cos\varphi,\quad y=r\sin\theta\sin\varphi,\quad z=r\cos\theta.
\]

\subsection{Скалярное произведение векторов}
Скалярным произведением двух векторов называется число, равное произведению длин векторов на косинус угла между ними:
\begin{equation}\label{skal}
\vec{a}\cdot\vec{b}=\left|\vec{a}\right|\cdot|\vec{b}|\cos\varphi,
\end{equation}
где $\varphi$ "--- угол между векторами $\vec{a}$ и $\vec{b}$.

Скалярное произведение $\vec{a}\cdot\vec{b}$ через координаты векторов $\vec{a}=\left\{a_x;a_y;a_z\right\}$ и $\vec{b}=\left\{b_x;b_y;b_z\right\}$ находится по формуле:
\begin{equation}\label{skal:koord}
\vec{a}\cdot\vec{b}=a_xb_x+a_yb_y+a_zb_z.
\end{equation}

\begin{center}
	\textbf{Свойства скалярного произведения}
\end{center}
\begin{enumerate}
	\item Переместительный закон: $\vec{a}\cdot\vec{b}=\vec{b}\cdot\vec{a}$.
	\item Сочетательный закон: $\left(\lambda\vec{a}\right)\cdot\vec{b}=\lambda\left(\vec{a}\cdot\vec{b}\right)$.
	\item Распределительный закон: $\vec{a}\cdot\left(\vec{b}+\vec{c}\right)=\vec{a}\cdot\vec{b}+\vec{a}\cdot\vec{c}$.
	\item Скалярный квадрат $\vec{a}^2$ вектора $\vec{a}$  равен квадрату его длины: $\left|\vec{a}\right|^2=\vec{a}^2$.
	\item Необходимым и достаточным условием перпендикулярности ненулевых векторов $\vec{a}$ и $\vec{b}$ является равенство нулю их скалярного произведения.
\end{enumerate}

\begin{center}
	\textbf{Основные приложения скалярного произведения}
\end{center}
\begin{enumerate}
	\item Вычисление работы $E$ постоянной (по модулю и направлению) силы $\vec{F}$  на пути  $\overrightarrow{AB}$: $E=\vec{F}\cdot\overrightarrow{AB}$.
	\item Вычисление угла $\varphi$ между векторами  $\vec{a}$ и $\vec{b}$: $\cos\varphi=\tfrac{\vec{a}\cdot\vec{b}}{\left|\vec{a}\right|\cdot|\vec{b}|}$ (см. формулу~(\ref{skal})).
	\item Вычисление проекции одного вектора на другой:
	\begin{equation}\label{eq:scal_proec}
	Пр_{\vec{a}}\vec{b}=\tfrac{\vec{a}\cdot\vec{b}}{\left|\vec{a}\right|}.
	\end{equation}
\end{enumerate}

\begin{examp}
	Даны $\left|\vec{a}\right|=3$, $|\vec{b}|=4$, $\varphi=\tfrac{2\pi}{3}$. Вычислить $\left(3\vec{a}-2\vec{b}\right)\cdot\left(\vec{a}+2\vec{b}\right)$.
	\par\textbf{Решение.}
	\begin{multline*}
	\left(3\vec{a}-2\vec{b}\right)\cdot\left(\vec{a}+2\vec{b}\right)=3\vec{a}^2+4\vec{a}\vec{b}-4\vec{b}^2=\\
	=3\left|\vec{a}\right|^2+4\left|\vec{a}\right|\cdot|\vec{b}|\cos\varphi-4|\vec{b}|^2=27-24-64=-61.
	\end{multline*}
\end{examp}

\begin{examp}
	Даны $\vec{a}=\left\{5;2;-1\right\}$, $\vec{b}=\left\{-1;3;3\right\}$. Проверить, являются ли эти векторы перпендикулярными.
	\par\textbf{Решение.}
	Найдём скалярное произведение векторов $\vec{a}$ и $\vec{b}$:
	\[
	\vec{a}\cdot\vec{b}=5\cdot(-1)+2\cdot 3+(-1)\cdot 3=-5+6-3=-2.
	\]
	Так как $\vec{a}\cdot\vec{b}=-2\neq 0$, то эти векторы не перпендикулярны.
\end{examp}

\subsection{Векторное произведение}
Векторным произведением вектора $\vec{a}$ на вектор $\vec{b}$ называется вектор $\vec{c}$, определяемый следующим образом (см. рис.~\ref{vekt}):
\begin{enumerate}
	\item $\vec{c}\perp\vec{a}$, $\vec{c}\perp\vec{b}$;
	\item $\left|\vec{c}\right|=\left|\vec{a}\right|\cdot|\vec{b}|\sin\varphi$, где $\varphi$ "--- угол между векторами $\vec{a}$ и $\vec{b}$;
	\item векторы $\vec{a}$, $\vec{b}$ и $\vec{c}$ образуют правую тройку, т.е. кратчайший поворот от вектора $\vec{a}$ к вектору $\vec{b}$, наблюдаемый с конца вектора $\vec{c}$, происходит против часовой стрелки.
\end{enumerate}
\begin{figure}[H]
	\ffigbox
	{\begin{tikzpicture}[thick,>=stealth']
		\draw[->] (0,0) -- node[below] {$\vec{a}$} (2.2,0);
		\draw[->] (0,0) -- node[left] {$\vec{c}$} (0,2);
		\draw[->] (0,0) -- node[above] {$\vec{b}$} (2,1);
		\draw[->] (1.3,0) to [controls=+(70:0.3) and +(-15:0.3)] node[right] {$\varphi$} (1,0.5);
		\draw (0,0.5) -- (0.5,0.5)--(0.5,0);
		\draw (0,0.5) -- (0.4472,0.7236)--(0.4472,0.2236);
		\end{tikzpicture}}
	{\caption{Определение векторного \\произведения векторов}\label{vekt}}
\end{figure}

Векторное произведение обозначается следующими символами $\vec{c}=\vec{a}\times\vec{b}=\left[\vec{a},\vec{b}\right]$.

Векторное произведение $\vec{a}\times\vec{b}$ через координаты векторов $\vec{a}=\left\{a_x;a_y;a_z\right\}$ и $\vec{b}=\left\{b_x;b_y;b_z\right\}$ можно записать в виде определителя:
\begin{equation}\label{vektproizv}
\vec{a}\times\vec{b}=\begin{vmatrix}
\vec{i} & \vec{j} & \vec{k}\\
a_x & a_y & a_z\\
b_x & b_y & b_z\\
\end{vmatrix}=\vec{i}\begin{vmatrix}
a_y & a_z\\
b_y & b_z\\
\end{vmatrix}-\vec{j}\begin{vmatrix}
a_x & a_z\\
b_x & b_z\\
\end{vmatrix}+\vec{k}\begin{vmatrix}
a_x & a_y\\
b_x & b_y\\
\end{vmatrix}.
\end{equation}

\begin{center}
	\textbf{Свойства и применения векторного произведения}
\end{center}
\begin{enumerate}
	\item Антипереместительный закон: $\vec{a}\times\vec{b}=-\vec{b}\times\vec{a}$.
	\item Сочетательный закон: $\left(\lambda\vec{a}\right)\times\vec{b}=\lambda\left(\vec{a}\times\vec{b}\right)$.
	\item Распределительный закон: $\vec{a}\times\left(\vec{b}+\vec{c}\right)=\vec{a}\times\vec{b}+\vec{a}\times\vec{c}$.
	\item Площадь параллелограмма, построенного на векторах $\vec{a}$ и $\vec{b}$ находится по формуле: $S_{\text{\pr}}=\left|\vec{a}\times\vec{b}\right|$; площадь треугольника:
	\begin{equation}\label{treug}
	S_{\text{\tr}}=\tfrac{1}{2}\left|\vec{a}\times\vec{b}\right|.
	\end{equation}
	\item Необходимым и достаточным условием коллинеарности ненулевых векторов $\vec{a}$ и $\vec{b}$ является равенство нулю их векторного произведения.
	\begin{figure}[H]
		\begin{floatrow}
			\ffigbox
			{\begin{tikzpicture}[thick,>=stealth']
				\draw[->] (1,1) -- (1,2.3);
				\draw[->] (1,1) -- (2,2);
				\draw[->,dashed] (0,1) -- (1,1);
				\draw (1,2.3) node[anchor=east] {$\overrightarrow{M}_0$};
				\draw (2,2) node[anchor=north] {$\vec{F}$};
				\draw[dashed] (-1,1) to [controls=+(90:0.5) and +(90:0.5)] (1,1);
				\draw (-1,1) to [controls=+(-90:0.5) and +(-90:0.5)] (1,1);
				\draw (0,1) circle(1);
				\fill (1,1) circle(1.5pt);
				\fill (0,1) circle(1.5pt);
				\draw (0,1) node[anchor=east] {O};
				\draw (1.2,1) node[anchor=north] {A};
				\end{tikzpicture}}
			{\caption{Момент силы $\vec{F}$, приложенной к точке А, относительно точки O}\label{moment}}
			\ffigbox
			{\begin{tikzpicture}[thick,>=stealth']
				\draw[->] (0,0) --  (2.2,0);
				\draw[->] (0,2) -- (0,3);
				\draw[->] (0,0) -- (-1.6,-0.8);
				\draw[dashed,->] (0,0) -- (0,1);
				\draw[dashed,->] (0,0) -- (0.5,0.68);
				\draw[dashed] (0,1) -- (0,2);
				\draw[dashed] (-1,1) to [controls=+(90:0.5) and +(90:0.5)] (1,1);
				\draw (-1,1) to [controls=+(-90:0.5) and +(-90:0.5)] (1,1);
				\draw (0,2.8) node[anchor=south east] {Z};
				\draw[->] (-0.3,2.5) to [controls=+(200:1) and +(-20:1)] (0.3,2.5);
				\draw (-1.9,-0.8) node[anchor=south] {X};
				\draw (2.2,0) node[anchor=south] {Y};
				\draw (0,1) node[anchor=east] {$\vec{\omega}$};
				\draw (0,0) node[anchor=north] {O};
				\draw (0,1) circle(1);
				\fill (0,0) circle(1.5pt);
				\fill (0.5,0.68) circle(1.5pt);
				\draw (0.5,0.68) node[anchor=south] {M};
				\draw[->] (0.5,0.68) --  (1.5,0.85);
				\draw (1.45,0.8) node[anchor=south] {$\vec{v}$};
				\end{tikzpicture}}
			{\caption{Линейная скорость $\vec{v}$ точки М твёрдого тела, вращающегося вокруг оси OZ}\label{skor}}
		\end{floatrow}
	\end{figure}
	\item Момент $\overrightarrow{M}_0$  силы $\vec{F}$, приложенной к точке А, относительно точки О вычисляется по формуле:
	\[
	\overrightarrow{M}_0=\vec{r}\times\vec{F},
	\]
	где $\vec{r}=\overrightarrow{OA}$  (рис.~\ref{moment}).
	\item Линейная скорость $\vec{v}$ точки М твёрдого тела, вращающегося вокруг неподвижной оси с угловой скоростью $\vec{\omega}$, также может быть вычислена с помощью векторного произведения:
	\[
	\vec{v}=\vec{\omega}\times\vec{r},
	\]
	где $\vec{r}=\overrightarrow{OM}$ "--- радиус-вектор точки M; вектор $\vec{\omega}$ направлен вдоль оси вращения (рис.~\ref{skor}).
\end{enumerate}



\begin{examp}
	Даны $\left|\vec{a}\right|=1$, $|\vec{b}|=3$, $\varphi=\tfrac{2\pi}{3}$. Найти $\left|\left(3\vec{a}-2\vec{b}\right)\times\left(\vec{a}+2\vec{b}\right)\right|$.
	\par\textbf{Решение.}
	\begin{multline*}
	\left|\left(3\vec{a}-2\vec{b}\right)\times\left(\vec{a}+2\vec{b}\right)\right|=\left|3\vec{a}\times\vec{a}+6\vec{a}\times\vec{b}
	-2\vec{b}\times\vec{a}-\right.\\ \left.-4\vec{b}\times\vec{b}\right|
	=\left|6\vec{a}\times\vec{b}+2\vec{a}\times\vec{b}\right|=8\left|\vec{a}\right|\cdot|\vec{b}|\sin\varphi=12\sqrt{3}.
	\end{multline*}
\end{examp}
\begin{examp}
	Сила $\vec{F}=\left\{1;1;1\right\}$ приложена к точке А$(0;2;-2)$. Найти момент этой силы относительно точки О$(2;-1;3)$.
	\par\textbf{Решение.}
	Так как $\overrightarrow{M}_0=\vec{r}\times\vec{F}$, где $\vec{r}=\overrightarrow{OA}=\left\{-2;3;-5\right\}$. Получаем:
	\[
	\overrightarrow{M}_0=\begin{vmatrix}
	\vec{i} & \vec{j} & \vec{k}\\
	-2 & 3 & -5\\
	1 & 1 & 1\\
	\end{vmatrix}=8\vec{i}-3\vec{j}-5\vec{k}.
	\]
\end{examp}


\subsection{Смешанное (векторно-скалярное) произведение трех векторов}
Смешанным (векторно-скалярным) произведением трех векторов $\vec{a}$, $\vec{b}$ и $\vec{c}$ называется произведение $\left(\vec{a}\times\vec{b}\right)\cdot\vec{c}$. Смешанное произведение обозначается символом $\vec{a\vphantom{b}}\,\vec{b}\,\vec{c\vphantom{b}}$.

Смешанное произведение $\vec{a\vphantom{b}}\,\vec{b}\,\vec{c\vphantom{b}}$ через координаты векторов, его образующих: $\vec{a}=\left\{a_x;a_y;a_z\right\}$, $\vec{b}=\left\{b_x;b_y;b_z\right\}$ и $\vec{c}=\left\{c_x;c_y;c_z\right\}$
можно записать в виде определителя:
\begin{equation}\label{smeshproizv}
\vec{a\vphantom{b}}\,\vec{b}\,\vec{c\vphantom{b}}=\begin{vmatrix}
a_x & a_y & a_z\\
b_x & b_y & b_z\\
c_x & c_y & c_z\\
\end{vmatrix}.
\end{equation}

\begin{center}
	\textbf{Свойства смешанного произведения}
\end{center}
\begin{enumerate}
	\item $\left(\vec{a}\times\vec{b}\right)\cdot\vec{c}=\vec{a}\cdot\left(\vec{b}\times\vec{c}\right)$.
	\item При перестановке в смешанном произведении двух соседних векторов его знак меняется на противоположный: $\vec{a\vphantom{b}}\,\vec{b}\,\vec{c\vphantom{b}}=-\vec{b}\,\vec{a\vphantom{b}}\,\vec{c\vphantom{b}}$.
	\item При круговой перестановке векторов смешанное произведение не меняется: $\vec{a\vphantom{b}}\,\vec{b}\,\vec{c\vphantom{b}}=\vec{c\vphantom{b}}\,\vec{a\vphantom{b}}\,\vec{b}=
	\vec{b}\,\vec{c\vphantom{b}}\,\vec{a\vphantom{b}}$.
	\item Объем параллелепипеда, построенного на векторах $\vec{a}$, $\vec{b}$ и $\vec{c}$, может быть вычислен с помощью смешанного произведения:  $V_{пар}=\left|\vec{a\vphantom{b}}\,\vec{b}\,\vec{c\vphantom{b}}\right|$. Отсюда следует, что объём треугольной пирамиды, построенной на тех же векторах, равен:
	\begin{equation}\label{ob_pir}
	V_{пир}=\tfrac{1}{6}\left|\vec{a\vphantom{b}}\,\vec{b}\,\vec{c\vphantom{b}}\right|.
	\end{equation}
	\item Необходимым и достаточным условием компланарности трех ненулевых векторов $\vec{a}$, $\vec{b}$ и $\vec{c}$ является равенство нулю их смешанного произведения: $\vec{a\vphantom{b}}\,\vec{b}\,\vec{c\vphantom{b}}=0$.
\end{enumerate}

\begin{examp}
	Даны  $\vec{a}=\left\{3;1;-1\right\}$, $\vec{b}=\left\{-1;0;3\right\}$  и  $\vec{c}=\left\{1;1;5\right\}$. Проверить, являются ли векторы компланарными.
	\par\textbf{Решение.}
	Найдем смешанное произведение векторов $\vec{a}$, $\vec{b}$ и $\vec{c}$ (см. формулу~(\ref{smeshproizv})).
	\begin{multline*}
	\vec{a\vphantom{b}}\,\vec{b}\,\vec{c\vphantom{b}}=\begin{vmatrix}
	3 & 1 & -1\\
	-1 & 0 & 3\\
	1 & 1 & 5\\
	\end{vmatrix}=3\cdot\left(0-3\right)-\\
	-1\cdot\left(-5-3\right)-1\cdot\left(-1-0\right)=-9+8+1=0.
	\end{multline*}
	Так как  $\vec{a\vphantom{b}}\,\vec{b}\,\vec{c\vphantom{b}}=0$, то векторы компланарны.
\end{examp}

\subsection{Линейное (векторное) пространство}
Совокупность $n$ действительных чисел $x_1,x_2,\ldots,x_n$, заданных в определённом порядке, называется $n$-мерным вектором. Сами числа $x_1,x_2,\ldots,x_n$ называются координатами вектора.

Пусть $\mathbf{x}=\{x_1;x_2;\ldots;x_n\}$ и $\mathbf{y}=\{y_1;y_2;\ldots;y_n\}$ --- два $n$-мерных вектора. Основными операциями над $n$-мерными векторами являются:
\begin{enumerate}
	\item Сложение: $\mathbf{x}+\mathbf{y}=\{x_1+y_1;x_2+y_2;\ldots;x_n+y_n\}$.
	\item Умножение на число: $\lambda\mathbf{x}=\{\lambda x_1;\lambda x_2;\ldots;\lambda x_n\}$.
\end{enumerate}

Векторы $\mathbf{x}$ и $\mathbf{y}$ называются равными, если равны их соответствующие координаты:
\[
\mathbf{x}=\mathbf{y}\quad\Leftrightarrow\quad x_1=y_1,x_2=y_2,\ldots,x_n=y_n.
\]

Вектор называется нулевым, если все его координаты равны нулю: $\mathbf{0}=\{0;0;\ldots;0\}$.

Вектор $-\mathbf{x}$ называется противоположным вектору $\mathbf{x}$, если $-\mathbf{x}=\{-x_1;-x_2;\ldots;-x_n\}$.

Операции сложения $n$-мерных векторов и умножения этих векторов на числа обладают свойствами:
\begin{enumerate}
	\item $\mathbf{x}+\mathbf{y}=\mathbf{y}+\mathbf{x}$;
	\item $\left(\mathbf{x}+\mathbf{y}\right)+\mathbf{z}=\mathbf{x}+\left(\mathbf{y}+\mathbf{z}\right)$;
	\item $\mathbf{x}+\mathbf{0}=\mathbf{x}$;
	\item $\mathbf{x}+\left(-\mathbf{x}\right)=\mathbf{0}$;
	\item $\lambda\left(\mu\mathbf{x}\right)=\left(\lambda\mu\right)\mathbf{x}$;
	\item $\lambda\left(\mathbf{x}+\mathbf{y}\right)=\lambda\mathbf{x}+\lambda\mathbf{y}$;
	\item $\left(\lambda+\mu\right)\mathbf{x}=\lambda\mathbf{x}+\mu\mathbf{x}$;
\end{enumerate}
где $\lambda,\mu$ --- произвольные вещественные числа, а $\mathbf{x},\mathbf{y},\mathbf{z}$ --- произвольные $n$-мерные векторы.

Множество всех $n$-мерных векторов, для которых установлены операции сложения и умножения на число, называется $n$-мерным векторным (линейным) пространством $\mathbf{R}^n$.

Система $n$-мерных векторов $\mathbf{x}_1,\mathbf{x}_2,\ldots,\mathbf{x}_s$ называется линейно зависимой, если существуют числа $\lambda_1,\lambda_2,\ldots,\lambda_s$, одновременно не равные нулю, такие, что
\[
\lambda_1\mathbf{x}_1+\lambda_2\mathbf{x}_2+\ldots+\lambda_s\mathbf{x}_s=0.
\]
В противном случае эта система векторов называется линейно независимой.

Пусть $Q$ --- произвольное подмножество $n$-мерных векторов пространства $\mathbf{R}^n$. Система векторов $\mathbf{B}=\{\mathbf{e}_1,\mathbf{e}_2,\ldots,\mathbf{e}_s\}$ называется базисом в $Q$, если выполняются следующие условия:
\begin{enumerate}
	\item $\mathbf{e}_k\in Q,\quad k=1,2,\ldots,s$;
	\item система $\mathbf{B}=\{\mathbf{e}_1,\mathbf{e}_2,\ldots,\mathbf{e}_s\}$ линейно независима;
	\item для любого вектора $\mathbf{x}\in Q$ найдутся числа $\lambda_1,\lambda_2,\ldots,\lambda_s$, такие, что $\mathbf{x}=\sum_{k=1}^s\lambda_k\mathbf{e}_k$.
\end{enumerate}
При этом формула $\mathbf{x}=\sum_{k=1}^s\lambda_k\mathbf{e}_k$ называется разложением вектора $\mathbf{x}$ по базису $\mathbf{B}$. Коэффициенты $\lambda_1,\lambda_2,\ldots,\lambda_s$ однозначно определяются вектором $\mathbf{x}$ и называются координатами этого вектора в базисе $\mathbf{B}$.

Если зафиксировать некоторый базис $\mathbf{B}=\{\mathbf{e}_1,\mathbf{e}_2,\ldots,\mathbf{e}_n\}$ в пространстве $\mathbf{R}^n$, тогда любому вектору $\mathbf{x}$ можно поставить во взаимно однозначное соответствие столбец его координат в этом базисе:
\[
\mathbf{x}=\sum_{k=1}^n\lambda_k\mathbf{e}_k=\begin{pmatrix}
\lambda_{1} \\
\lambda_{2} \\
\vdots \\
\lambda_{n} \\
\end{pmatrix}.
\]

\subsection{Линейные операторы. Собственные значения и собственные векторы}
Линейным оператором в линейном $n$-мерном пространстве $\mathbf{R}^n$ называется отображение $A:\mathbf{R}^n\rightarrow\mathbf{R}^n$ пространства $\mathbf{R}^n$ в себя, обладающее свойствами:
\begin{enumerate}
	\item $A\left(\lambda\mathbf{x}\right)=\lambda A\mathbf{x}$;
	\item $A\left(\mathbf{x}+\mathbf{y}\right)=A\mathbf{x}+A\mathbf{y}$.
\end{enumerate}

Если $A$ --- линейный оператор в $\mathbf{R}^n$ и $\mathbf{B}=\{\mathbf{e}_1,\mathbf{e}_2,\ldots,\mathbf{e}_n\}$ --- фиксированный базис, то векторы $A\mathbf{e}_k$ могут быть разложены по базису $\mathbf{B}$:
\begin{equation}\label{aek}
A\mathbf{e}_k=\sum_{j=1}^n a_{jk}\mathbf{e}_j=\begin{pmatrix}
a_{1k} \\
a_{2k} \\
\vdots \\
a_{nk} \\
\end{pmatrix},\quad k=1,2,\ldots,n.
\end{equation}
Матрица коэффициентов этого разложения
\begin{equation}\label{liop}
A = \begin{pmatrix}
a_{11} & a_{12} & \cdots & a_{1n} \\
a_{21} & a_{22} & \cdots & a_{2n} \\
\vdots & \vdots & \ddots & \vdots \\
a_{n1} & a_{n2} & \cdots & a_{nn} \\
\end{pmatrix}.
\end{equation}
называется матрицей линейного оператора $A$ в базисе $\mathbf{B}$. Столбцы этой матрицы в точности равны вектор-столбцам~(\ref{aek}).

Если число $\lambda$ и вектор $\mathbf{x}$ являются решениями уравнения
\begin{equation}\label{sob}
A\mathbf{x}=\lambda\mathbf{x},
\end{equation}
то число $\lambda$ называется собственным значением линейного оператора $A$, а вектор $\mathbf{x}$ --- собственным вектором линейного оператора $A$, отвечающим собственному значению $\lambda$. Поскольку при фиксированном базисе в $\mathbf{R}^n$ линейный оператор полностью определяется своей матрицей~(\ref{liop}), то допустимо говорить, что число $\lambda$ и вектор $\mathbf{x}$ являются собственным значением и собственным вектором матрицы линейного оператора. Более того, в линейном (векторном) пространстве $\mathbf{R}^n$ равенство~(\ref{sob}) эквивалентно матричному уравнению
\begin{equation}\label{matrur}
\left(A-\lambda E\right)\mathbf{x}=0.
\end{equation}
Это уравнение, как следует из~(\ref{eq:systitog_matr}), имеет ненулевое решение только если
\begin{equation}\label{sobzn}
\det\left(A-\lambda E\right)=0,
\end{equation}
т.е. если не существует обратная матрица $\left(A-\lambda E\right)^{-1}$.

Уравнение~(\ref{sobzn}) представляет собой алгебраическое уравнение $n$-ой степени относительно $\lambda$. Его решениями являются все собственные значения оператора $A$. Для отыскания собственных векторов $\mathbf{x}$, соответствующих найденным из~(\ref{sobzn}) собственным значениям $\lambda$, необходимо при этих $\lambda$, найти ненулевые решения системы линейных однородных алгебраических уравнений~(\ref{matrur}).

Пример нахождения собственных значений и собственных векторов квадратной матрицы 2-го порядка приведён на стр.~\pageref{sob2}.

%\begin{examp}
%Найти собственные значения матрицы $A=\begin{pmatrix}
%  3 & -8\\
%  2 & -7\\
%\end{pmatrix}.$
%\par\textbf{Решение.}
%Составим и решим уравнение на собственные значения (см.~(\ref{sobzn})):
%\[
%\det\left(\begin{pmatrix}
%  3 & -8\\
%  2 & -7\\
%\end{pmatrix}-\lambda\begin{pmatrix}
%  1 & 0\\
%  0 & 1\\
%\end{pmatrix}\right)=0\quad\Rightarrow\quad\begin{vmatrix}
%  3-\lambda & -8\\
%  2 & -7-\lambda\\
%\end{vmatrix}=0
%\]
%\[
%\left(3-\lambda\right)\left(-7-\lambda\right)+16=0\,\Rightarrow\,\lambda^2+4\lambda-5=0\,\Rightarrow\,\lambda_1=-5,\lambda_2=1.
%\]
%Собственные значения матрицы $A$: $\lambda_1=-5,\lambda_2=1$.
%\end{examp}

\newpage
\section{Примеры решения типовых задач}
\noindent\hrulefill\,\fbox{\textbf{Задание № 1}}\,\hrulefill\vspace{4pt}\\
Найти решение $\left(x_0;y_0;z_0\right)$ системы $\left\{ \begin{gathered}
2x - 3y + 5z = 12 \hfill \\
x + 2y - z = -1 \hfill \\
5x + 6y - 3z =  1 \hfill \\
\end{gathered}  \right.$ и вычислить $2x_0+3z_0-y_0$.\vspace{4pt}

\noindent\fbox{\textbf{Решение}}\,\hrulefill\vspace{4pt}

Решим систему уравнений тремя способами: методом Крамера, методом Гаусса и матричным методом.
\begin{center}
	\textbf{Метод Крамера}
\end{center}

Вычислим главный определитель системы $\Delta$. Для этого, используя формулу~(\ref{glopr}), заполним коэффициентами перед неизвестными $x,y,z$, соответственно, первый, второй и третий столбцы определителя:
\begin{multline*}
\Delta = \begin{vmatrix}
2 & -3 & 5 \\
1 & 2 & -1 \\
5 & 6 & -3 \\
\end{vmatrix}=\left\{\text{\small{раскладываем по первой строке}}\right\}=\\
=2\cdot(-1)^2\begin{vmatrix}
2 & -1 \\
6 & -3 \\
\end{vmatrix}-3\cdot(-1)^3\begin{vmatrix}
1 & -1 \\
5 & -3 \\
\end{vmatrix}+5\cdot(-1)^4\begin{vmatrix}
1 & 2 \\
5 & 6 \\
\end{vmatrix}=\\
=\left\{\text{\small{определители 2-го порядка вычислим}}\right.\\
\left.\text{\small{с помощью формулы~(\ref{eq:opr2})}}\right\}=
\end{multline*}
\begin{multline*}
=2\cdot\left(2\cdot(-3)-(-1)\cdot6\right)+3\cdot\left(1\cdot(-3)-(-1)\cdot5\right)+
\\+5\cdot\left(1\cdot6-2\cdot5\right)=
\\=2\cdot(-6+6)+3\cdot(-3+5)+5\cdot(6-10)=0+6-20=-14.
\end{multline*}

Далее вычислим определитель $\Delta_1$ неизвестной $x$, который получается из главного определителя $\Delta$ путём замены первого столбца столбцом свободных членов:
\begin{multline*}
\Delta_1 = \begin{vmatrix}
12 & -3 & 5 \\
-1 & 2 & -1 \\
1 & 6 & -3 \\
\end{vmatrix}=\left\{\text{\small{раскладываем по первой строке}}\right\}=\\
=12\cdot(-1)^2\begin{vmatrix}
2 & -1 \\
6 & -3 \\
\end{vmatrix}-3\cdot(-1)^3\begin{vmatrix}
-1 & -1 \\
1 & -3 \\
\end{vmatrix}+5\cdot(-1)^4\begin{vmatrix}
-1 & 2 \\
1 & 6 \\
\end{vmatrix}=\\
=12\cdot(-6+6)+3\cdot(3+1)+5\cdot(-6-2)=0+12-40=-28.
\end{multline*}
Следовательно, по формулам Крамера~(\ref{eq:kram}), неизвестная $x$ равна: $x=\tfrac{\Delta_1}{\Delta}=\tfrac{-28}{-14}=2$.

Затем, вычислим определитель $\Delta_2$ неизвестной $y$, который получается из главного определителя $\Delta$ путём замены второго столбца столбцом свободных членов:
\begin{multline*}
\Delta_2 = \begin{vmatrix}
2 & 12 & 5 \\
1 & -1 & -1 \\
5 & 1 & -3 \\
\end{vmatrix}=2\begin{vmatrix}
-1 & -1 \\
1 & -3 \\
\end{vmatrix}-12\begin{vmatrix}
1 & -1 \\
5 & -3 \\
\end{vmatrix}+5\begin{vmatrix}
1 & -1 \\
5 & 1 \\
\end{vmatrix}=\\
=2\cdot(3+1)-12\cdot(-3+5)+5\cdot(1+5)=8-24+30=14.
\end{multline*}
Следовательно, по формулам Крамера~(\ref{eq:kram}), неизвестная $y$ равна: $y=\tfrac{\Delta_2}{\Delta}=\tfrac{14}{-14}=-1$.

Аналогично находим неизвестную $z$:
\begin{multline*}
\Delta_3 = \begin{vmatrix}
2 & -3 & 12 \\
1 & 2 & -1 \\
5 & 6 & 1 \\
\end{vmatrix}=2\begin{vmatrix}
2 & -1 \\
6 & 1 \\
\end{vmatrix}+3\begin{vmatrix}
1 & -1 \\
5 & 1 \\
\end{vmatrix}+12\begin{vmatrix}
1 & 2 \\
5 & 6 \\
\end{vmatrix}=\\
=2\cdot(2+6)+3\cdot(1+5)+12\cdot(6-10)=16+18-48=-14.
\end{multline*}
Неизвестная $z$ равна: $z=\tfrac{\Delta_3}{\Delta}=\tfrac{-14}{-14}=1$.

Таким образом, решение системы имеет вид: $\left(2;-1;1\right)$.

\begin{center}
	\textbf{Метод Гаусса}
\end{center}

Выпишем расширенную матрицу системы, используя формулу~(\ref{eq:matr}):
\begin{equation*}
\left(A/B\right)=\left(\begin{matrix}
2 & -3 & 5 \\
1 & 2 & -1 \\
5 & 6 & -3 \\
\end{matrix}\right.\left|\begin{matrix}
12 \\
-1 \\
1 \\
\end{matrix}\right).
\end{equation*}
Преобразуем полученную матрицу к верхнетреугольному виду: под главной диагональю элементы матрицы равны нулю. Для
удобства преобразования элемент $a_{11}$ (стоящий в первой строке и первом столбце) сделаем равным единице. Этого
можно добиться, разделив все элементы первой строки на два, тогда $a_{11}=1$, но при этом в строке появятся дробные
числа, что усложняет дальнейшие преобразования. Поэтому воспользуемся другим элементарным преобразованием расширенной матрицы:
поменяем местами первую и вторую строки:
\begin{equation*}
\left(A/B\right)\sim\left(\begin{matrix}
1 & 2 & -1 \\
2 & -3 & 5 \\
5 & 6 & -3 \\
\end{matrix}\right.\left|\begin{matrix}
-1 \\
12 \\
1 \\
\end{matrix}\right).
\end{equation*}
В первом столбце под элементом $a_{11}=1$ нужно получить ноль. Для этого умножим все элементы первой строки на $(-2)$ (элемент
стоящий во второй строке  первом столбце, взятый с обратным знаком) и прибавим их к соответствующим элементам второй строки:
\begin{equation*}\small
\left(\begin{matrix}
1 & 2 & -1 \\
2+1\cdot(-2) & -3+2\cdot(-2) & 5+(-1)\cdot(-2) \\
5 & 6 & -3 \\
\end{matrix}\right.\left|\begin{matrix}
-1 \\
12+(-1)\cdot(-2) \\
1 \\
\end{matrix}\right)=
\end{equation*}
\begin{equation*}
=\left(\begin{matrix}
1 & 2 & -1 \\
0 & -7 & 7 \\
5 & 6 & -3 \\
\end{matrix}\right.\left|\begin{matrix}
-1 \\
14 \\
1 \\
\end{matrix}\right).
\end{equation*}

Для получения нуля на месте элемента $a_{13}=5$, умножим все элементы первой строки на $(-5)$ и прибавим к соответствующим
элементам третьей строки:
\begin{equation*}\small
\left(\begin{matrix}
1 & 2 & -1 \\
0 & -7 & 7 \\
5+1\cdot(-5) & 6+2\cdot(-5) & -3+(-1)\cdot(-5) \\
\end{matrix}\right.\left|\begin{matrix}
-1 \\
14 \\
1+(-1)\cdot(-5) \\
\end{matrix}\right)=
\end{equation*}
\begin{equation*}
=\left(\begin{matrix}
1 & 2 & -1 \\
0 & -7 & 7 \\
0 & -4 & 2 \\
\end{matrix}\right.\left|\begin{matrix}
-1 \\
14 \\
6 \\
\end{matrix}\right).
\end{equation*}

Под главной диагональю остался один элемент, не равный нулю: $a_{32}=-4$. Для того, чтобы получить в этой позиции ноль, вначале создадим единицу на
месте элемента $a_{22}=-7$. С этой целью разделим все элементы второй строки на $(-7)$:

\begin{equation*}
\left(A/B\right)\sim\left(\begin{matrix}
1 & 2 & -1 \\
0 & 1 & -1 \\
0 & -4 & 2 \\
\end{matrix}\right.\left|\begin{matrix}
-1 \\
-2 \\
6 \\
\end{matrix}\right).
\end{equation*}
Далее умножим все элементы второй строки на $4$ (элемент, стоящий под элементом $a_{22}$ на пересечении третьей строки и второго столбца,
взятый с обратным знаком) и прибавим к соответствующим элементам третьей строки:
\begin{equation*}\footnotesize
\left(A/B\right)\sim\left(\begin{matrix}
1 & 2 & -1 \\
0 & 1 & -1 \\
0 & -4+1\cdot4 & 2+(-1)\cdot4 \\
\end{matrix}\right.\left|\begin{matrix}
-1 \\
-2 \\
6-2\cdot4 \\
\end{matrix}\right)
=\left(\begin{matrix}
1 & 2 & -1 \\
0 & 1 & -1 \\
0 & 0 & -2 \\
\end{matrix}\right.\left|\begin{matrix}
-1 \\
-2 \\
-2 \\
\end{matrix}\right).
\end{equation*}
Получили верхнетреугольную матрицу.

Применим обратный ход метода Гаусса для нахождения не\-известных. Последней строке верхнетреугольной матрицы соответствует уравнение:
$-2z=-2$.

\emph{Все элементы верхнетреугольной матрицы "--- это коэффициенты при неизвестных: элементы первого столбца "--- коэффициенты при $x$,
	элементы второго столбца "--- при $y$, третьего "--- при $z$. Таким образом, число $a_{33}=-2$ является коэффициентом при неизвестной $z$.
	Вертикальная черта символизирует знак равенства.}

Из этого уравнения находим: $z=\tfrac{-2}{-2}=1$.

Второй строке верхнетреугольной матрицы соответствует уравнение: $1\cdot y-1\cdot z=-2$. Подставим в него найденное значение $z=1$:
$1\cdot y-1\cdot 1=-2$. Отсюда: $y=-2+1=-1$.

Наконец, первой строке верхнетреугольной матрицы соответствует уравнение: $1\cdot x+2\cdot y-1\cdot z=-1$. Подставим в него вместо $y$ и $z$
найденные значения: $1\cdot x+2\cdot(-1)-1\cdot1=-1$. Откуда: $x=-1+2+1=2$.

Таким образом, решение системы имеет вид: $\left(2;-1;1\right)$.

\begin{center}
	\textbf{Матричный метод}
\end{center}\vspace*{-0.5cm}

Главная матрица системы имеет вид:
$A=\begin{pmatrix}
2 & -3 & 5 \\
1 & 2 & -1 \\
5 & 6 & -3\\
\end{pmatrix}$. Её определитель, как мы выяснили при использовании метода Крамера, не равен нулю: $\det A =\Delta=-14$.
Следовательно, обратная матрица $A^{-1}$ существует. Для её нахождения вычислим алгебраические дополнения элементов определителя $\det A$:
\begin{equation*}
A_{11}=(-1)^2\begin{vmatrix}
2 & -1 \\
6 & -3 \\
\end{vmatrix}=0;\quad A_{12}=(-1)^3\begin{vmatrix}
1 & -1 \\
5 & -3 \\
\end{vmatrix}=-2;
\end{equation*}
\begin{equation*}
A_{13}=(-1)^4\begin{vmatrix}
1 & 2 \\
5 & 6 \\
\end{vmatrix}=-4;\quad A_{21}=(-1)^3\begin{vmatrix}
-3 & 5 \\
6 & -3 \\
\end{vmatrix}=21;
\end{equation*}
\begin{equation*}
A_{22}=(-1)^4\begin{vmatrix}
2 & 5\\
5 & -3 \\
\end{vmatrix}=-3;\quad A_{23}=(-1)^5\begin{vmatrix}
2 & -3 \\
5 & 6 \\
\end{vmatrix}=-27;
\end{equation*}
\begin{equation*}
A_{31}=(-1)^4\begin{vmatrix}
-3& 5\\
2& -1 \\
\end{vmatrix}=-7;\quad A_{32}=(-1)^5\begin{vmatrix}
2 & 5 \\
1 & -1 \\
\end{vmatrix}=7;
\end{equation*}
\begin{equation*}
A_{33}=(-1)^6\begin{vmatrix}
2 & -3 \\
1 & 2 \\
\end{vmatrix}=7.
\end{equation*}
Теперь по формуле~(\ref{eq:obrmatr}) находим обратную матрицу:
\begin{equation*}
A^{-1}=\frac{1}{-14}\begin{pmatrix}
0 & 21 & -7 \\
-2 & -31 & 7 \\
-4 & -27 & 7\\
\end{pmatrix}=\begin{pmatrix}
0 & -\tfrac{3}{2} & \tfrac{1}{2} \\
\tfrac{1}{7}\vphantom{\sqrt{\tfrac{1}{6}}} & \tfrac{31}{14} & -\tfrac{1}{2} \\
\tfrac{2}{7}\vphantom{\sqrt{\tfrac{1}{6}}} & \tfrac{27}{14} & -\tfrac{1}{2}\\
\end{pmatrix}.
\end{equation*}
Поскольку в данном примере
$B=\begin{pmatrix}
12 \\
-1 \\
1 \\
\end{pmatrix}$, то, применяя формулы~(\ref{eq:systitog_matr_resh}) и~(\ref{eq:proizvmatr}), найдём решение системы уравнений:
\begin{equation*}
X=\begin{pmatrix}
0 & -\tfrac{3}{2} & \tfrac{1}{2} \\
\tfrac{1}{7}\vphantom{\sqrt{\tfrac{1}{6}}} & \tfrac{31}{14} & -\tfrac{1}{2} \\
\tfrac{2}{7}\vphantom{\sqrt{\tfrac{1}{6}}} & \tfrac{27}{14} & -\tfrac{1}{2}\\
\end{pmatrix}\cdot\begin{pmatrix}
12 \\
-1 \\
1 \\
\end{pmatrix}=\begin{pmatrix}
0 + \tfrac{3}{2} + \tfrac{1}{2}  \\
\tfrac{12}{7}\vphantom{\sqrt{\tfrac{1}{6}}}-\tfrac{31}{14}-\tfrac{1}{2} \\
\tfrac{24}{7}\vphantom{\sqrt{\tfrac{1}{6}}}-\tfrac{27}{14}-\tfrac{1}{2} \\
\end{pmatrix}=\begin{pmatrix}
2\\
-1 \\
1\\
\end{pmatrix}.
\end{equation*}

Таким образом, решение системы имеет вид: $\left(2;-1;1\right)$.

Остаётся выполнить само тестовое задание, т.е. найти значение выражения $2x_0+3z_0-y_0$. Так как $x_0=2$, $y_0=-1$, $z_0=1$, то
получаем $2x_0+3z_0-y_0=2\cdot2+3\cdot1-(-1)=8$.

\noindent\textbf{Ответ: }8.\vspace{4pt}

\noindent\hrulefill\,\fbox{\textbf{Задание № 2}}\,\hrulefill\\

\vspace*{-0.5cm}\noindent\fbox{\textbf{\small{Вариант № 1}}}\vspace{4pt}\\
Даны матрицы $A=\begin{pmatrix}
2 & -1 & 3\\
4 & 5 & 1\\
\end{pmatrix}$ и $B=\begin{pmatrix}
3 & 0 & 2\\
-1 & 7 & 1\\
\end{pmatrix}$. Найти $2A+3B$.

\noindent\fbox{\textbf{Решение}}\,\hrulefill\vspace{4pt}

Умножим все элементы матрицы $A$ на 2, а все элементы матрицы $B$ "--- на 3.
\begin{equation*}
2A=\begin{pmatrix}
4 & -2 & 6\\
8 & 10 & 2\\
\end{pmatrix},\quad
3B=\begin{pmatrix}
9 & 0 & 6\\
-3 & 21 & 3\\
\end{pmatrix}.
\end{equation*}
Найдём сумму матриц $2A$ и $3B$. Для этого прибавим к элементам матрицы $2A$ соответствующие элементы матрицы
$3B$:
\begin{equation*}
2A+3B=\begin{pmatrix}
4+9 & -2+0 & 6+6\\
8+(-3) & 10+21 & 2+3\\
\end{pmatrix}=
\begin{pmatrix}
13 & -2 & 12\\
5 & 31 & 5\\
\end{pmatrix}.
\end{equation*}
\noindent\textbf{Ответ: }$\begin{pmatrix}
13 & -2 & 12\\
5 & 31 & 5\\
\end{pmatrix}$.\vspace{4pt}

\noindent\fbox{\textbf{\small{Вариант № 2}}}\vspace*{-0.5cm}\\
Даны матрицы $A=\begin{pmatrix}
3 & -1 & 6\\
5 & 4 & 2\\
\end{pmatrix}$ и $B=\begin{pmatrix}
5 & 1\\
2 & 0\\
-1 & 3\\
\end{pmatrix}$. Найти $A\!\cdot\! B$.

\noindent\fbox{\textbf{Решение}}\,\hrulefill\vspace{4pt}

Новая матрица $A\!\cdot\! B$ будет иметь столько же строк сколько и матрица $A$ и столько же столбцов сколько имеет матрица $B$.
Таким образом, в нашем случае, матрица $A\!\cdot\! B$ состоит из 2-х строк и 2-х столбцов.

Для получения элемента, стоящего на пересечении первой строки и первого столбца матрицы $A\!\cdot\! B$, найдём сумму произведений
элементов первой строки матрицы $A$ на соответствующие элементы первого столбца матрицы $B$.

Для получения элемента, стоящего на пересечении первой строки и второго столбца матрицы $A\!\cdot\! B$, найдём сумму произведений
элементов первой строки матрицы $A$ на соответствующие элементы второго столбца матрицы $B$.

Аналогично находим оставшиеся элементы (см. формулу~(\ref{eq:proizvmatr})):
\begin{equation*}
A\!\cdot\! B=\begin{pmatrix}
3 & -1 & 6\\
5 & 4 & 2\\
\end{pmatrix}\cdot\begin{pmatrix}
5 & 1\\
2 & 0\\
-1 & 3\\
\end{pmatrix}=
\end{equation*}
\begin{equation*}
=\begin{pmatrix}
3\cdot5+(-1)\cdot2+6\cdot(-1) & 3\cdot1+(-1)\cdot0+6\cdot3\\
5\cdot5+4\cdot2+2\cdot(-1) & 5\cdot1+4\cdot0+2\cdot3\\
\end{pmatrix}=\begin{pmatrix}
7 & 21\\
31 & 11\\
\end{pmatrix}.
\end{equation*}
\noindent\textbf{Ответ: }$\begin{pmatrix}
7 & 21\\
31 & 11\\
\end{pmatrix}$.\vspace{4pt}

\noindent\hrulefill\,\fbox{\textbf{Задание № 3}}\,\hrulefill\vspace{4pt}\\
Найти элемент матрицы, обратной к $A=\begin{pmatrix}
3 & -5 & 6\\
1 & 0 & 4\\
5 & -3 & 1\\
\end{pmatrix}$, расположенный на пересечении третьей строки и второго столбца.\vspace{4pt}

\noindent\fbox{\textbf{Решение}}\,\hrulefill\vspace{4pt}

Обратную матрицу можно найти только для невырожденных квадратных матриц, т.е. для таких квадратных матриц, определитель которых не равен нулю:

\begin{multline*}
\det A=\begin{vmatrix}
3 & -5 & 6\\
1 & 0 & 4\\
5 & -3 & 1\\
\end{vmatrix}=3\cdot\begin{vmatrix}
0 & 4\\
-3 & 1\\
\end{vmatrix}-(-5)\cdot\begin{vmatrix}
1 &  4\\
5 & 1\\
\end{vmatrix}
+6\cdot\begin{vmatrix}
0 & 4\\
-3 & 1\\
\end{vmatrix}=\\
=3(0+12)+5(1-20)+6(-3-0)=36-95-18=-77.
\end{multline*}
Так как $\det A\neq0$, следовательно, обратная матрица $A^{-1}$ существует.

Найдём элемент обратной матрицы, расположенный на пересечении третьей строки и второго столбца. Для этого воспользуемся формулой (см.~(\ref{eq:obrmatr})):
\begin{equation*}
\left(A^{-1}\right)_{32}=\frac{A_{23}}{\det A},
\end{equation*}
где $A_{23}$ "--- алгебраическое дополнение элемента $a_{23}$ матрицы $A$:

\begin{equation*}
A_{23}=(-1)^{2+3}\begin{vmatrix}
3 & -5\\
5 & -3\\
\end{vmatrix}=-(-9+25)=-16.
\end{equation*}

Таким образом, искомый элемент обратной матрицы равен: \[\left(A^{-1}\right)_{32}=\frac{-16}{-77}=\frac{16}{77}.\]
\noindent\textbf{Ответ: }$\frac{16}{77}$.\vspace{4pt}

\noindent\hrulefill\,\fbox{\textbf{Задание № 4}}\,\hrulefill\vspace{4pt}\\
Найти разложение вектора $\vec{c}=\{7;-4\}$ по базису векторов $\vec{a}=\{3;2\}$ и $\vec{b}=\{5;-1\}$.\vspace{4pt}

\noindent\fbox{\textbf{Решение}}\,\hrulefill\vspace{4pt}

Разложить вектор  $\vec{c}$ по базису $\vec{a}$, $\vec{b}$ --- это значит представить его в виде: $\vec{c}=\alpha\vec{a}+\beta\vec{b}$. Для нахождения неизвестных коэффициентов $\alpha$ и $\beta$ подставим в эту формулу координаты векторов $\vec{a}$, $\vec{b}$ и $\vec{c}$, а затем выполним действия над векторами в правой части:
\begin{equation*}
\{7;-4\}=\alpha\{3;2\}+\beta\{5;-1\};
\end{equation*}
\begin{equation*}
\{7;-4\}=\{3\alpha;2\alpha\}+\{5\beta;-\beta\};
\end{equation*}
\begin{equation*}
\{7;-4\}=\{3\alpha+5\beta;2\alpha-\beta\}.
\end{equation*}
Два вектора равны, если равны их соответствующие координаты. Следовательно, справедлива система уравнений:
\begin{equation*}
\left\{ \begin{gathered}
3\alpha+5\beta = 7 \hfill \\
2\alpha-\beta = -4 \hfill \\
\end{gathered}  \right..
\end{equation*}
Решим её методом Крамера:
\begin{equation*}
\Delta = \begin{vmatrix}
3 & 5 \\
2 & -1 \\
\end{vmatrix}=3\cdot(-1)-2\cdot5=-13;
\end{equation*}
\begin{equation*}
\Delta_1 = \begin{vmatrix}
7 & 5 \\
-4 & -1 \\
\end{vmatrix}=13;\Delta_2 = \begin{vmatrix}
3 & 7 \\
2 & -4 \\
\end{vmatrix}=-26.
\end{equation*}
Отсюда:
$
\alpha = \frac{\Delta_1}{\Delta}=\frac{13}{-13}=-1;\quad\beta=\frac{\Delta_2}{\Delta}=\frac{-26}{-13}=2.
$
Остаётся подставить найденные коэффициенты $\alpha$ и $\beta$ в разложение вектора $\vec{c}$: \[\vec{c}=\alpha\vec{a}+\beta\vec{b}=-\vec{a}+2\vec{b}.\]
\noindent\textbf{Ответ: }$\vec{c}=-\vec{a}+2\vec{b}$.\vspace{4pt}


\noindent\hrulefill\,\fbox{\textbf{Задание № 5}}\,\hrulefill\vspace{4pt}\\
Найти координаты и длину вектора $\vec{c}=2\vec{a}-3\vec{b}$, где $\vec{a}=\{6;-1;3\}$, $\vec{b}=\{4;0;3\}$.\vspace{4pt}

\noindent\fbox{\textbf{Решение}}\,\hrulefill\vspace{4pt}

Для того, чтобы найти координаты вектора $\vec{c}$, вначале определим координаты векторов $2\vec{a}$ и $3\vec{b}$:
$2\vec{a}=\{12;-2;6\}$, $3\vec{b}=\{12;0;9\}$, а затем вычтем из координат вектора $2\vec{a}$ соответствующие координаты вектора $3\vec{b}$:
\[
\vec{c}=\{12;-2;6\}-\{12;0;9\}=\{12-12;-2-0;6-9\}=\{0;-2;-3\}.
\]
Теперь, с помощью формулы~(\ref{eq:dlin_vec}), найдём длину вектора $\vec{c}$:
\[
\left|\vec{c}\,\right|=\sqrt{0^2+(-2)^2+(-3)^2\mathstrut}=\sqrt{4+9\mathstrut}=\sqrt{13\mathstrut}.
\]

\noindent\textbf{Ответ: }$\vec{c}=\{0;-2;-3\}$, $\left|\vec{c}\,\right|=\sqrt{13\mathstrut}$.\vspace{4pt}

\noindent\hrulefill\,\fbox{\textbf{Задание № 6}}\,\hrulefill\vspace{1pt}\\

\noindent\fbox{\textbf{\small{Вариант № 1}}}\vspace{4pt}\\
Найти проекцию вектора $\vec{a}=2\vec{i}-4\vec{j}+\vec{k}$ на вектор $\vec{b}=3\vec{i}+\vec{j}-2\vec{k}$.\vspace{4pt}

\noindent\fbox{\textbf{Решение}}\,\hrulefill\vspace{4pt}

Для решения используем формулу~(\ref{eq:scal_proec}): $Пр_{\vec{b}}\vec{a}=\tfrac{\vec{a}\cdot\vec{b}}{\left|\vec{b}\right|}$.

Найдём скалярное произведение векторов $\vec{a}\cdot\vec{b}$, используя формулу~(\ref{skal:koord}):
$
\vec{a}\cdot\vec{b}=2\cdot3+(-4)\cdot1+1\cdot(-2)=0.
$
Так как скалярное произведение равно нулю, то векторы $\vec{a}$ и $\vec{b}$ перпендикулярны и, следовательно, $Пр_{\vec{b}}\vec{a}=0$.

\noindent\textbf{Ответ: }$Пр_{\vec{b}}\vec{a}=0$.\vspace{4pt}

\noindent\fbox{\textbf{\small{Вариант № 2}}}\vspace{4pt}\\
Пусть $\left|\vec{a}\right|=4$, $|\vec{b}\,|=1$, угол между векторами $\vec{a}$ и $\vec{b}$ равен $\varphi=\tfrac{\pi}{6}$. Найти значение выражения $\left|(2\vec{a}-\vec{b}\,)\times(2\vec{b}+3\vec{a}\,)\right|$.

\noindent\fbox{\textbf{Решение}}\,\hrulefill\vspace{4pt}

Раскроем скобки и воспользуемся следующими свойствами векторного произведения: \[\vec{a}\times\vec{b}=-\vec{b}\times\vec{a},\, \vec{a}\times\vec{a}=0,\,
\left|\vec{a}\times\vec{b}\right|=\left|\vec{a}\right|\cdot|\vec{b}|\sin\varphi.\]
\begin{equation*}
\left|(2\vec{a}-\vec{b}\,)\times(2\vec{b}+3\vec{a}\,)\right|=
\left|2\vec{a}\times2\vec{b}+2\vec{a}\times3\vec{a}-\vec{b}\times2\vec{b}
-\vec{b}\times3\vec{a}\right|=
\end{equation*}
\begin{equation*}
=\left|4\vec{a}\times\vec{b}+0-0+3\vec{a}\times\vec{b}\right|=\left|7\vec{a}\times\vec{b}\right|=
\end{equation*}
\begin{equation*}
=7\left|\vec{a}\right|\cdot|\vec{b}|\sin\varphi=7\cdot4\cdot1\cdot\tfrac{1}{2}=14.
\end{equation*}

\noindent\textbf{Ответ: }14.\vspace{4pt}


\noindent\hrulefill\,\fbox{\textbf{Задание № 7}}\,\hrulefill\\

\vspace*{-0.5cm}\noindent\fbox{\textbf{\small{Вариант № 1}}}\vspace{4pt}\\
Найти площадь треугольника $ABC$, если $A(2;0;4)$, $B(-3;-1;2)$, \\$C(5;-3;0)$.

\noindent\fbox{\textbf{Решение}}\,\hrulefill\vspace{4pt}

С помощью~(\ref{conec_nachalo}) найдём координаты векторов $\overrightarrow{AB}$ и $\overrightarrow{AC}$:
\[
\overrightarrow{AB}=\left\{-3-2;-1-0;2-4\right\}=\left\{-5;-1;-2\right\};
\]
\[
\overrightarrow{AC}=\left\{5-2;-3-0;0-4\right\}=\left\{3;-3;-4\right\}.
\]

Площадь треугольника $ABC$ вычисляется по формуле (см.~(\ref{treug})): $S_{\text{\tr}}=\tfrac{1}{2}\left|\overrightarrow{AB}\times\overrightarrow{AC}\right|$.
Найдём векторное произведение векторов $\overrightarrow{AB}$ и $\overrightarrow{AC}$ по формуле~(см.~(\ref{vektproizv})):
\begin{multline*}
\overrightarrow{AB}\times\overrightarrow{AC}=\begin{vmatrix}
\vec{i} & \vec{j} & \vec{k}\\
-5 & -1 & -2\\
3 & -3 & -4\\
\end{vmatrix}=\vec{i}\begin{vmatrix}
-1 & -2\\
-3 & -4\\
\end{vmatrix}-\\-\vec{j}\begin{vmatrix}
-5 & -2\\
3 & -4\\
\end{vmatrix}+\vec{k}\begin{vmatrix}
-5 & -1\\
3 & -3\\
\end{vmatrix}=-2\vec{i}-26\vec{j}+18\vec{k}.
\end{multline*}
Теперь вычислим длину вектора $\overrightarrow{AB}\times\overrightarrow{AC}$:
\begin{multline*}
\left|\overrightarrow{AB}\times\overrightarrow{AC}\right|=\sqrt{(-2)^2+(-26)^2+18^2\mathstrut}=\\=\sqrt{4+676+324\mathstrut}=\sqrt{1004\mathstrut}=
2\sqrt{251\mathstrut}.
\end{multline*}
Отсюда:
$
S_{\text{\tr}}=\tfrac{1}{2}\cdot2\sqrt{251\mathstrut}=\sqrt{251\mathstrut}\quad(\text{кв. ед.}).
$

\noindent\textbf{Ответ: }$S_{\text{\tr}}=\sqrt{251\mathstrut}\quad(\text{кв. ед.})$.\vspace{4pt}

\noindent\fbox{\textbf{\small{Вариант № 2}}}\vspace{4pt}\\
Найти координаты вектора $(\vec{a}+\vec{b}\,)\times(2\vec{a}-3\vec{b}\,)$, если $\vec{a}=\{1;0;-1\}$, $\vec{b}=\{3;-2;4\}$.

\noindent\fbox{\textbf{Решение}}\,\hrulefill\vspace{4pt}

Найдём координаты векторов $\vec{a}+\vec{b}$ и $2\vec{a}-3\vec{b}$:
\[
\vec{a}+\vec{b}=\{1;0;-1\}+\{3;-2;4\}=\{4;-2;3\};
\]
\begin{multline*}
2\vec{a}-3\vec{b}=2\{1;0;-1\}-3\{3;-2;4\}=\\=\{2;0;-2\}-\{9;-6;12\}=\{-7;6;-14\}.
\end{multline*}
Вновь воспользуемся формулой~(\ref{vektproizv}):
\begin{multline*}
(\vec{a}+\vec{b}\,)\times(2\vec{a}-3\vec{b}\,)=\begin{vmatrix}
\vec{i} & \vec{j} & \vec{k}\\
4 & -2 & 3\\
-7 & 6 & -14\\
\end{vmatrix}=\vec{i}\begin{vmatrix}
-2 & 3\\
6 & -14\\
\end{vmatrix}-\\-\vec{j}\begin{vmatrix}
4 & 3\\
-7 & -14\\
\end{vmatrix}+\vec{k}\begin{vmatrix}
4 & -2\\
-7 & 6\\
\end{vmatrix}=10\vec{i}+35\vec{j}+10\vec{k}.
\end{multline*}
Таким образом, $(\vec{a}+\vec{b}\,)\times(2\vec{a}-3\vec{b}\,)=\{10;35;10\}$.

\noindent\textbf{Ответ: }$\{10;35;10\}$.\vspace{4pt}

\noindent\hrulefill\,\fbox{\textbf{Задание № 8}}\,\hrulefill\vspace{4pt}\\
Найти объём треугольной пирамиды $ABCD$, если $A(0;1;1)$, $B(2;-1;4)$, $C(3;2;0)$, $D(-1;0;0)$.

\noindent\fbox{\textbf{Решение}}\,\hrulefill\vspace{4pt}

Найдём координаты векторов $\overrightarrow{AB}$, $\overrightarrow{AC}$ и $\overrightarrow{AD}$. Воспользуемся формулой~(\ref{conec_nachalo}):
\[
\overrightarrow{AB}=\left\{2-0;-1-1;4-1\right\}=\left\{2;-2;3\right\};
\]
\[
\overrightarrow{AC}=\left\{3-0;2-1;0-1\right\}=\left\{3;1;-1\right\};
\]
\[
\overrightarrow{AD}=\left\{-1-0;0-1;0-1\right\}=\left\{-1;-1;-1\right\}.
\]

Объём треугольной пирамиды $ABCD$ вычисляется по формуле (см.~(\ref{ob_pir})): $V_{\text{пир}}=\tfrac{1}{6}\left|\overrightarrow{AB}\,\overrightarrow{AC}\,\overrightarrow{AD}\right|$. Находим смешанное произведение векторов $\overrightarrow{AB}$, $\overrightarrow{AC}$ и $\overrightarrow{AD}$ (см.~(\ref{smeshproizv})):
\begin{multline*}
\overrightarrow{AB}\,\overrightarrow{AC}\,\overrightarrow{AD}=\begin{vmatrix}
2 & -2 & 3\\
3 & 1 & -1\\
-1 & -1 & -1\\
\end{vmatrix}=2\begin{vmatrix}
1 & -1\\
-1 & -1\\
\end{vmatrix}-\\-(-2)\begin{vmatrix}
3 & -1\\
-1 & -1\\
\end{vmatrix}+3\begin{vmatrix}
3 & 1\\
-1 & -1\\
\end{vmatrix}=-4-8-6=-18.
\end{multline*}
Отсюда, $V_{\text{пир}}=\tfrac{1}{6}\left|-18\right|=3\quad(\text{куб. ед.}).$

\noindent\textbf{Ответ: }$V_{\text{пир}}=3\quad(\text{куб. ед.})$.\vspace{4pt}

\noindent\hrulefill\,\fbox{\textbf{Задание №9}}\,\hrulefill\vspace{4pt}\label{sob2}\\
Найти собственные значения и собственные векторы матрицы
\[A=\begin{pmatrix}
-3 & 5\\
2 & 6\\
\end{pmatrix}.
\]

\noindent\fbox{\textbf{Решение}}\,\hrulefill\vspace{4pt}

Уравнение на собственные значения имеет вид (см.~(\ref{sobzn})):
\begin{equation*}\small
\det\left(\begin{pmatrix}
-3 & 5\\
2 & 6\\
\end{pmatrix}-\lambda\begin{pmatrix}
1 & 0\\
0 & 1\\
\end{pmatrix}\right)=0\quad\Rightarrow\quad\begin{vmatrix}
-3-\lambda & 5\\
2 & 6-\lambda\\
\end{vmatrix}=0;
\end{equation*}
\[
\left(-3-\lambda\right)\left(6-\lambda\right)-10=0\,\Rightarrow\,\lambda^2-3\lambda-28=0\,\Rightarrow\,\lambda_1=-4,\lambda_2=7.
\]
Собственные значения матрицы $A$: $\lambda_1=-4,\lambda_2=7$.

Вычислим координаты собственных векторов, отвечающих найденным собственным значениям.

\textbf{1)} При $\lambda=-4$ (см.~(\ref{matrur})):
\begin{equation*}\footnotesize
\left(\begin{pmatrix}
-3 & 5\\
2 & 6\\
\end{pmatrix}+4\cdot\begin{pmatrix}
1 & 0\\
0 & 1\\
\end{pmatrix}\right)\begin{pmatrix}
x_1 \\
x_2\\
\end{pmatrix}=\begin{pmatrix}
0 \\
0\\
\end{pmatrix}\Rightarrow\begin{pmatrix}
1 & 5\\
2 & 10\\
\end{pmatrix}\begin{pmatrix}
x_1 \\
x_2\\
\end{pmatrix}=\begin{pmatrix}
0 \\
0\\
\end{pmatrix}.
\end{equation*}
Данное матричное уравнение эквивалентно системе двух линейных однородных алгебраических уравнений:
\[
\left\{ \begin{gathered}
x_1 + 5x_2 = 0 \hfill \\
2x_1 + 10x_2 = 0 \hfill \\
\end{gathered}  \right.\,\Rightarrow\,\left\{ \begin{gathered}
x_1 + 5x_2 = 0 \hfill \\
x_1 + 5x_2 = 0 \hfill \\
\end{gathered}  \right. .
\]
Таким образом, $x_1=-5x_2$. Положим $x_2=c_1$, где $c_1$ --- произвольная постоянная. Тогда $x_1=-5с_1$ и, следовательно, собственный вектор $\mathbf{x_1}$, отвечающий собственному значению $\lambda_1=-4$, имеет вид:
$
\mathbf{x_1}=\{-5c_1;c_1\}=c_1\{-5;1\}$ или $\mathbf{x_1}=c_1\begin{pmatrix}
-5 \\
1\\
\end{pmatrix}.
$
\textbf{2)} При $\lambda=7$ действуем аналогично:
\begin{equation*}\footnotesize
\left(\begin{pmatrix}
-3 & 5\\
2 & 6\\
\end{pmatrix}-7\cdot\begin{pmatrix}
1 & 0\\
0 & 1\\
\end{pmatrix}\right)\begin{pmatrix}
x_1 \\
x_2\\
\end{pmatrix}=\begin{pmatrix}
0 \\
0\\
\end{pmatrix}\Rightarrow\left\{ \begin{gathered}
-10x_1 + 5x_2 = 0 \hfill \\
2x_1 - x_2 = 0 \hfill \\
\end{gathered}  \right..
\end{equation*}
%Данное матричное уравнение эквивалентно системе двух линейных однородных алгебраических уравнений:
%\[
%\left\{ \begin{gathered}
%  -10x_1 + 5x_2 = 0 \hfill \\
%  2x_1 - x_2 = 0 \hfill \\
%\end{gathered}  \right.\,\Rightarrow\,\left\{ \begin{gathered}
%  2x_1 - x_2 = 0 \hfill \\
%  2x_1 - x_2 = 0 \hfill \\
%\end{gathered}  \right. .
%\]
Таким образом, $x_2=2x_1$. Положим $x_1=c_2$, где $c_2$ --- произвольная постоянная. Тогда $x_2=2с_2$ и, следовательно, собственный вектор $\mathbf{x_2}$, отвечающий собственному значению $\lambda_2=7$, имеет вид:
$
\mathbf{x_2}=\{c_2;2c_2\}=c_2\{1;2\}$ или $\mathbf{x_2}=c_2\begin{pmatrix}
1 \\
2\\
\end{pmatrix}.
$

\noindent\textbf{Ответ: }$\lambda_1=-4,\,\mathbf{x_1}=c_1\begin{pmatrix}
-5 \\
1\\
\end{pmatrix};\lambda_2=7,\,\mathbf{x_2}=c_2\begin{pmatrix}
1 \\
2\\
\end{pmatrix}$.\vspace{4pt}


%\section{Векторная алгебра}
%\section{Векторы и линейные операции над ними}
%\begin{defnt}
%Вектором называется направленный отрезок прямой.
%\end{defnt}
%Если точка $A$ --- начало вектора, а $B$ --- его конец, то вектор обозначают символом $\overrightarrow{AB}$.
%Можно также использовать строчные буквы латинского алфавита, например, $\vec{a}$. Длина вектора
%обозначается символом $|\overrightarrow{AB}|$ или $\left|\vec{a}\right|$ и равна
%расстоянию между началом и концом вектора.
%
%\begin{defnt}
%Вектор называется нулевым, если его начало и конец совпадают.
%\end{defnt}
%Таким образом, нулевой вектор $\overrightarrow{O}$ имеет длину, равную нулю, и не имеет определённого направления.
%
%\begin{defnt}
%Векторы, лежащие на параллельных прямых, называются коллинеарными.
%Векторы, лежащие в параллельных плоскостях, называются компланарными.
%\end{defnt}
%
%Вектор $\vec{a}$ равен вектору $\vec{b}$, т.е. $\vec{a}=\vec{b}$, если выполняются следующие три условия:
%\begin{enumerate}
%  \item $\left|\vec{a}\right|=|\vec{b}|$;
%  \item $\vec{a}\parallel\vec{b}$ (векторы коллинеарны);
%  \item $\vec{a}\upuparrows\vec{b}$ (векторы сонаправлены).
%\end{enumerate}
%
%\section{Прямоугольная система координат. Координаты вектора}
%
%\section{Скалярное произведение векторов}
%
%\begin{thebibliography}{99}
%\bibitem{Ilin:2002} \emph{Ильин В.\,А., Куркина А.\,В.} Высшая
%математика. -- М.: ООО <<ТК Велби>>, 2002. --- 592 с.
%\bibitem{lit2}
%\bibitem{lit3}
%\bibitem{lit4}
%\end{thebibliography}


\label{lastpage}
\TooltipHidden
\end{document}