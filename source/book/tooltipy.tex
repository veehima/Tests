\documentclass[10pt]{article}
\usepackage[papersize={5in,5in},margin=1pt]{geometry}
\usepackage{tikz}
\usetikzlibrary{shapes,snakes}
\usepackage{fancybox,xspace,colortbl,calc,ifthen}
\usepackage{settobox}
\usepackage[utf8]{inputenc}
\usepackage[russian]{babel}
\usepackage{amsmath,dejavu}
\usepackage[pdftex,unicode,pdfborder={0 0 0},pdfhighlight=/P]{hyperref}
\usepackage{flowfram}
\usepackage{picins}
\usepackage[createtips]{fancytooltips}
\usepackage{multido}

%\pdfmapfile{DejaVuSans.map}


\newcommand{\sym}[2]{{\fontencoding{U}\fontfamily{#1}\selectfont\symbol{#2}}}

\newenvironment{zametka}[5]%
{\begin{center}\fboxsep=1.6\fboxrule \shadowsize=4pt\begin{Sbox}
\begin{minipage}[c]{#4}
\keytip{#3}\parpic[r][t]{\includegraphics[width=#5]{#1}}
\picskip{#2}\small}%
{\end{minipage}\end{Sbox}\shadowbox{\fboxsep=5pt\colorbox[rgb]{1,0.988,0.737}{\TheSbox}}
\end{center}}

\tikzstyle{mybox} = [draw=blue, fill=green!20, very thick,
    rectangle, rounded corners, inner sep=10pt, inner ysep=20pt]
\tikzstyle{fancytitle} =[fill=blue, text=white, ellipse]
%



\begin{document}

\begin{zametka}{silvestr}{9}{silvestr}{0.9\textwidth}{0.25\textwidth}
Джеймс Джозеф (Sylvester James Joseph), род. 03.09.1814, в Лондоне,
ум. 15.03.1897 в Лондоне. Английский математик, c 08.12.1872
иностранный чл.--корр. Петербургской АН: физико-математическое
отделение (по разряду математических наук). В 1837 окончил
Кембриджский университет, профессор Королевской академии в Вулидже
(1855-1870), с 1876 г. по 1883 г. --- университета Джонса Хопкинса в
Балтиморе (США), с 1833 г. --- Оксфордского университета. Основные
труды  --- в области алгебры (в частности, по теории инвариантов),
теории чисел, теории вероятностей, механики и математической физики.
Основал (1878) первый американский математический журнал <<The
American Journal of Mathematics>>.
\end{zametka}
\newpage
\begin{zametka}{cayley}{9}{cayley}{0.9\textwidth}{0.25\textwidth}
Кэли (Кейли) Артур (Cayley Arthur), род. 16.8.1821 в Ричмонде -- ум.
26.1.1895 в  Кембридже. Английский математик, иностранный чл.--корр.
Петербургской АН (c 04.12.1870), член Лондонского королевского
общества (1852), с 1863 г. профессор Кембриджского университета.
Основные работы по теории алгебраических квадратичных форм.
Установил связь между теорией инвариантов и проективной геометрией.
Исследования Кэли в этой области легли в основу истолкования
геометрии Лобачевского (<<интерпретация Кэли--Клейна>>).
Автор работ по теории определителей (ввел в 1841 г. общепринятое
ныне обозначение для определителя), дифференциальных уравнений,
эллиптических функций. Занимался также сферической астрономией и
астрофизикой.
\end{zametka}
\newpage
\begin{zametka}{cochy}{9}{cochy}{0.9\textwidth}{0.25\textwidth}
Коши Огюстен Луи (Cauchy Augustin Louis), род. 21.8.1789, Париж --
ум. 23.5.1857, Со.
\\
Французский математик, иностранный почётный член Петербургской АН (с
14.12.1831), чл. Парижской АН (1816). Окончил Политехническую школу
(1807) и Школу мостов и дорог (1810) в Париже. В 1810--13 работал
инженером в Шербуре. В 1816--30 преподавал в Политехнической школе и
в Колледж де Франс, с 1848 --- в Парижском университете и в Колледж
де Франс.
\\
Труды Коши относятся к различным областям математики. Его курсы
математического анализа (<<Курс анализа>>, 1821, <<Резюме
лекций по исчислению бесконечно малых>>, 1823, <<Лекции по
приложениям анализа к геометрии>>, т. 1--2, 1826--28), основанные
на систематическом использовании понятия предела, послужили образцом
для большинства курсов позднейшего времени. В них он дал определение
понятия непрерывности функции, чёткое построение теории сходящихся
рядов (признак Коши, критерий Коши), определение интеграла как
предела сумм и др. В теории аналитических функций комплексного
переменного Коши дал выражение аналитической функции в виде т.н.
интеграла Коши, разработал теорию вычетов. В области теории
дифференциальных уравнений Коши принадлежат: постановка т.н. задачи
Коши, основные теоремы существования решений и метод интегрирования
уравнений с частными производными 1-го порядка.
\end{zametka}
\newpage
\begin{zametka}{note}{2}{commut}{0.5\textwidth}{0.15\textwidth}
{\bf Коммутативность} от латинского \emph{commutatius} ---
меняющийся, переставляющий.
\end{zametka}
\newpage
\begin{zametka}{note}{2}{assot}{0.5\textwidth}{0.15\textwidth}
{\bf Ассоциативность} от латинского
 \emph{assotiatio} --- соединение.
\end{zametka}
\newpage
\begin{zametka}{note}{2}{distrib}{0.5\textwidth}{0.15\textwidth}
{\bf Дистрибутивность} от латинского
    \emph{distributivus} --- распределительный.
\end{zametka}
\newpage
\begin{zametka}{leibn}{11}{leibn}{0.92\textwidth}{0.25\textwidth}
Лейбниц Готфрид Вильгельм (Leibniz Gottfried Wilhelm), род.
1.7.1646, Лейпциг -- ум. 14.11.1716, Ганновер.
\\
Немецкий философ-идеалист, математик, физик и изобретатель, юрист,
историк, языковед, член Лондонского королевского общества (1673),
член Парижской АН (1700). Изучал юриспруденцию и философию в
Лейпцигском и Йенском университетах. В 1672 отправился с
дипломатической миссией в Париж, где пробыл до 1676, изучая
математику и естествознание. В декабре 1676 возвратился в Германию и
последующие 40 лет состоял на службе у ганноверских герцогов,
сначала в качестве придворного библиотекаря, затем --- герцогского
историографа и тайного советника юстиции. В 1700 стал первым
президентом созданного по его инициативе Бранденбургского научного
общества (позднее --- Берлинская АН). В 1711, 1712 и 1716 встречался
с Петром I, разработал по его просьбе ряд проектов по развитию
образования и государственного управления в России.
\\
В математике важнейшей заслугой Лейбница является разработка (наряду
с И.Ньютоном) дифференциального и интегрального исчислений, в
которых он правильно увидел важнейший инструмент для разработки
проблем физики.
\\
Кроме анализа Лейбниц сделал ряд важных открытий  в комбинаторике,
алгебре (начала теории определителей), геометрии.
\end{zametka}
\newpage
\begin{center}
\begin{tikzpicture}[transform shape, rotate=0, baseline=-3.5cm]
\node [mybox] (box) {%
    \keytip{ya}\begin{minipage}[t!]{0.7\textwidth}
        Михеев Андрей Вячеславович --- кандидат физико-математических наук,
        доцент кафедры высшей математики
        Казанского национального исследовательского технологического университета.
    \end{minipage}
    };
\node[fancytitle] at (box.north) {Об авторе};
\end{tikzpicture}
\end{center}
\newpage
\begin{zametka}{Cramer_2}{9}{kramer}{0.9\textwidth}{0.25\textwidth}
Крамер Габриель (31.7.1704--04.01.1752) --- швейцарский математик. Родился в Женеве. Был учеником и другом Иоганна Бернулли. Издатель трудов Иоганна и Якова Бернулли, переписки Г. Лейбница с И. Бернулли. Учился и работал в Женеве.
\\
Основные труды по высшей алгебре и аналитической геометрии. Установил и опубликовал (1750 г.) правила решения систем n линейных уравнений с n неизвестными с буквенными коэффициентами (правило Крамера), заложил основы теории определителей, но при этом еще не пользовался удобным обозначением определителей. Во <<Введении в анализ алгебраических кривых>> (1750 г.) существенно развил идеи современников по аналитической геометрии; исследовал особые точки, ветви, кривизну алгебраических кривых высших порядков. В 1742 г. Крамер обобщил на случай трех неподвижных точек поставленную еще Паппом задачу о вписании в круг треугольника, стороны которого проходят через три точки, лежащие на одной прямой. В геометрии известен парадокс Крамера. Член Лондонского королевского общества (1749 г.)
\end{zametka}




%\def\obrazek#1{
%  \begin{flushright}
%    \color{red}
%    \fboxsep 0 pt{\shadowbox{{\color{black}\includegraphics[width=0.8\hsize,
%        page=#1,viewport= 0 57 350 230,clip]{tecna2.pdf}}}}
%  \end{flushright}
%\newpage}
%
%
%\multido{\i=1+1}{25}{\obrazek{\i}}

\end{document}